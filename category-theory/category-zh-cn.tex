\ifx\wholebook\relax \else

\documentclass{article}

\input{../common-zh-cn.tex}

\setcounter{page}{1}

\begin{document}

\title{范畴论}

\author{刘新宇
\thanks{{\bfseries 刘新宇} \newline
  Email: liuxinyu95@gmail.com \newline}
  }

\maketitle
\fi

\markboth{范畴论}{编程的数学原理}

\ifx\wholebook\relax
\chapter{范畴论}
\numberwithin{Exercise}{chapter}
\fi

\epigraph{数学是赋予不同事物相同名字的艺术。}{——昂利$\cdot$庞加莱}

% Mathematics is the art of giving the same name to different things.

如果你已经坚持看到了本书这一章,我建议你小小的奖励自己一下。你已经迈过了第一道门槛,正在通往神奇的抽象王国之路上。这条路是世界上许多最聪明的心智披荆斩棘开辟出来的。如果说,人们将具体的事物,抽象成不带具体意义的数与形是原始阶段;将数、形与计算的意义去除,抽象成代数结构(例如群)和代数关系(例如同构)是第一阶段;范畴论可以算是抽象的第二阶段。

你也许会问,我们为什么要了解范畴论?这和编程有什么关系?对此有一个比较短的答案和一个比较长的答案。较短的回答是,如果不了解范畴论,过不了多久,你也许看不懂别人写的程序了。2010年以后,如果翻看Haskell标准库的源代码,就会发现几乎所有的内容,都用范畴论重新写过了。映射、叠加、遍历……几乎所有的计算都在说着范畴的语言,犹如天书。你也许觉得叠加操作可以这样写:

\lstset{language=Haskell, frame=single}
 \begin{lstlisting}
foldr _ z [] = z
foldr f z (x:xs) = f x (foldr f z xs)
\end{lstlisting}

实际上,今天的标准库用范畴的语言这样写:

\begin{lstlisting}
foldr f z t = appEndo (foldMap (Endo #. f) t) z
foldMap f = foldr (mappend . f) mempty
\end{lstlisting}

你也许觉得,反正工作中不用Haskell,不了解范畴论也没有关系。但是最近十几年的情况是,范畴论由于其强大的抽象,几乎普适于任何问题,正向其他的语言和环境中渗透。不要说各大编程语言纷纷引入lambda演算和闭包等结构,有超过20种语言已经实现了单子\cite{Monad-Haskell-Wiki}——用范畴论的语言说,叫做“函子范畴上的幺半群”。

较长的答案是:我们需要抽象。请原谅这个答案看起来更短。赫尔曼$\cdot$外尔说现代数学在过去几十年不断沉湎在抽象和形式化上。编程领域何尝不是如此呢?现代计算机科学解决的问题空前复杂,大数据量、分布式、高并发、还要保证数据和计算的安全。仅仅靠着前几十年的传统方法——暴力求解、务实的工程实践再加上一点聪明的头脑已经不够了。这逼迫着我们去吸取其他科学和数学中的新方法和新工具。

正如迪厄多内所说:“这种抽象绝不是来自数学家的反常意愿,似乎他们想通过使用深奥莫测的语言来把自己与其它人隔开。数学家是被经典对象和关系的本质特性逼着去锻造新的抽象工具,来解决过去看来是不可攻克的问题。”\cite{Dieudonne1987}

本章内容将再次挑战我们的抽象思维。如果读完一变没有理解是完全正常的。请不要灰心丧气,生活不是直线发展的,我们理解知识的过程是螺旋上升的。

塞缪尔·艾伦伯格(Samuel Eilenberg,1913年9月30日-1998年1月30日)是一个波兰-美国数学家,犹太血统。他出生于俄罗斯帝国时期的华沙(现在为波兰),逝于美国纽约市,他是纽约哥伦比亚大学的教授,在那里度过了大部分职业生涯。

他于1936年在华沙大学获得哲学博士学位。他的论文导师是Karol Borsuk。他主要研究兴趣是代数拓扑。他与诺曼·斯廷罗德一起建立了同调论的公理化,与桑德斯·麦克兰恩(Saunders Mac Lane)合作公理化了同调代数。在这个过程中,艾伦伯格与 Mac Lane 创立了范畴论。

艾伦伯格加入了尼古拉·布尔巴基小组,与昂利·嘉当合作,1956年著有《同调代数 Homological Algebra》,这是一部经典著作。

其后他主要工作是纯粹范畴论,是该领域的奠基者之一。Eilenberg swindle(或 telescope)是将裂项消元法想法运用于投射模的一个构造。

1986年获得沃尔夫奖。

\begin{wrapfigure}{R}{0.3\textwidth}
 \centering
 \includegraphics[scale=0.25]{img/Eilenberg.eps}
 \captionsetup{labelformat=empty}
 \caption{艾伦伯格(Samuel Eilenberg, 1913 - 1998)}
 \label{fig:Pythagoras}
\end{wrapfigure}

桑德斯·麦克兰恩Saunders Mac Lane(1909/08/04,康乃狄克州~2005/04/14旧金山)是一位美国数学家。与赛谬尔·艾伦伯格一同创立范畴论的研究。

桑德斯·麦克兰恩的正式全名为"雷斯礼·桑德斯·麦克兰恩",但雷斯礼这名字最后因他的父亲,堂诺得·麦克兰恩,及其母亲,维妮福瑞德·桑德斯的不喜欢而不被再使用。关于麦克兰恩(Mac Lane)中间是否要加一空白这问题,麦克兰恩的第一任妻子在打字时输入麦克兰恩时,发现不要插入空白非常困难,因此麦克兰恩自己就开始在他的姓中间插入了一个空白。麦克兰恩于1930年获得耶鲁大学学士学位,在1931年获得芝加哥大学的硕士学位,在这段时期,他与欧文·兰米尔合作,写出他生平第一篇论文,内容主要关于物理学方面。麦克兰恩于1931~1933年间至哥廷根大学进修,投入Paul Bernays、Emmy Noether与Hermann Weyl的门下研习逻辑与数学,并逾1934年获得哥廷根数学研究所数学博士学位。
自1934年至1938年,麦克兰恩分别在哈佛大学、康乃尔大学以及芝加哥大学工作过一小段时间,随后取得哈佛大学的终身职,但他仅于1938~1947年在哈佛大学工作,他的学术生涯接下来都在芝加哥大学度过。
在1944年与1945年,他领导了在大战期间有卓越贡献的哥伦比亚大学应用数学小组。
麦克兰恩曾任国家科学院与美国哲学会的副主席,亦曾担任过美国数学会的主席。在领导美国数学会的其间,他开始提倡对于现代数学教学技巧改进的研习活动。而在1974年至1980年间,他担任了美国政府的科学顾问。于1976年,以他为首的美国数学家访问团访问了中国,考察了当时中国数学学术发展。麦克兰恩于1949年获选为美国国家科学院院士,并在1989年获得美国国家科学奖章。与塞缪尔·艾伦伯格合作公理化了同调代数。在这个过程中,艾伦伯格与 Mac Lane 创立了范畴论。

\begin{wrapfigure}{R}{0.3\textwidth}
 \centering
 \includegraphics[scale=1]{img/Mac-Lane.eps}
 \captionsetup{labelformat=empty}
 \caption{麦克莱恩(Saunders Mac Lane, 1909 - 2005)}
 \label{fig:Pythagoras}
\end{wrapfigure}

在发表了关于数学逻辑论文后,麦克兰恩的早期研究方向为域论与赋值论,他在赋值环、维特向量(Witt vector)以及无限扩张域的可分离性发表了许多论文。从1942年开始,他开始研究了扩张群,并逾1943年,与塞谬尔·艾伦伯格展开了一项化时代性的伟大研究——现在称为“范畴论”的重要研究领域。

\ifx\wholebook\relax \else
\begin{thebibliography}{99}

\bibitem{Dieudonne1987}
[法]让$\cdot$迪厄多内 著,沈用欢 译 ``当代数学,为了人类心智的荣耀''. 上海教育出版社. 2000年3月. ISBN: 7532063062

\bibitem{Monad-Haskell-Wiki}
Haskell Wiki. ``Monad''. \url{https://wiki.haskell.org/Monad}

\end{thebibliography}

\expandafter\enddocument
%\end{document}

\fi
