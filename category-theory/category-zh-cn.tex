\ifx\wholebook\relax \else

\documentclass{article}

\input{../common-zh-cn.tex}

\setcounter{page}{1}

\begin{document}

\title{范畴论}

\author{刘新宇
\thanks{{\bfseries 刘新宇} \newline
  Email: liuxinyu95@gmail.com \newline}
  }

\maketitle
\fi

\markboth{范畴论}{编程的数学原理}

\ifx\wholebook\relax
\chapter{范畴论}
\numberwithin{Exercise}{chapter}
\fi

\epigraph{你只要能把自己提出的那些“点、线、面”都说的跟“桌子、椅子、啤酒杯子”一样自然连贯就行。}{——大卫$\cdot$希尔伯特}

\ifx\wholebook\relax \else
\begin{thebibliography}{99}

\bibitem{HanXueTao16}
韩雪涛 ``数学悖论与三次数学危机''. 人民邮电出版社. 2016, ISBN: 9787115430434

\end{thebibliography}

\expandafter\enddocument
%\end{document}

\fi
