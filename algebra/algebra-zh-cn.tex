\ifx\wholebook\relax \else

\documentclass[UTF8]{article}

\input{../common-zh-cn.tex}

\setcounter{page}{1}

\begin{document}

\title{群、环、域}

\author{刘新宇
\thanks{{\bfseries 刘新宇} \newline
  Email: liuxinyu95@gmail.com \newline}
  }

\maketitle
\fi

\markboth{群、环、域}{编程的数学原理}

\ifx\wholebook\relax
\chapter{群、环、域}
\numberwithin{Exercise}{chapter}
\fi

\epigraph{你只要能把自己提出的那些“点、线、面”,都说的跟“桌子、椅子、啤酒杯子”一样自然而连贯就行}{——大卫$\cdot$希尔伯特}

\begin{wrapfigure}{R}{0.3\textwidth}
 \centering
 \includegraphics[scale=0.4]{img/Escher-Liberation-1955.eps}
 \caption{艾舍尔《解放》}
 \label{fig:clay-token}
\end{wrapfigure}

我们人类在长期的生产生活中逐渐养成了对事物分类整理的习惯。不同但相近的东西被归为一类。对整个类适用的性质和方法,对类中的不同事物都有效。这样我们就从解决具体的单一的问题,提高到一下子解决整类的抽象的问题,极大地丰富了我们认识掌握世界的能力。

在第一章中,我们曾经从自然数的累加和阶乘中归纳抽象出了“叠加”操作。让我们再回顾一下。

累加$0 + 1 + 2 + ... $的定义:

\be
\begin{array}{l}
sum(0) = 0 \\
sum(n + 1) = (n + 1) + sum(n)
\end{array}
\ee

阶乘$n!$的定义:

\be
\begin{array}{l}
fact(0) = 1 \\
fact(n + 1) = (n + 1) \cdot fact(n)
\end{array}
\ee

在计算机程序中,人们可以定义一种叫做“二叉搜索树”的数据结构。在第二章中,我们曾经介绍过二叉树的定义。二叉搜索树是一种特殊的二叉树,它要求树中元素的类型A是可以比较的\footnote{这里比较的含义是抽象的,例如可以比大小,也可以是比较单词在字典中的先后顺序},并且对于任意分支节点,节点左侧子树中的所有元素都在此节点中元素之前;而右子树中的所有元素都在此元素之后。对这样

\section{群}

\section{环}

\section{域}

\ifx\wholebook\relax \else
\begin{thebibliography}{99}

\bibitem{HanXueTao16}
韩雪涛 ``数学悖论与三次数学危机''. 人民邮电出版社. 2016, ISBN: 9787115430434

\end{thebibliography}

\expandafter\enddocument
%\end{document}

\fi
