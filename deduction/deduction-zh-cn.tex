\ifx\wholebook\relax \else

\documentclass{article}

\input{../common-zh-cn.tex}

\setcounter{page}{1}

\begin{document}

\title{推理}

\author{刘新宇
\thanks{{\bfseries 刘新宇} \newline
  Email: liuxinyu95@gmail.com \newline}
  }

\maketitle
\fi

\markboth{推理}{编程的数学原理}

\ifx\wholebook\relax
\chapter{推理}
\numberwithin{Exercise}{chapter}
\fi

\epigraph{3表示2+1,4表示3+1。所以接下来(虽然证明很长),4等于2+2。因此, 数学知识不再是神秘的。}{——罗素}

% ... mathematical knowledge ... is, in fact, merely verbal knowledge. "3" means "2+1", and "4" means "3+1". Hence it follows (though the proof is long) that "4" means the same as "2+2". Thus mathematical knowledge ceases to be mysterious.  -- Bertrand Russell

\begin{wrapfigure}{R}{0.4\textwidth}
 \centering
 \includegraphics[scale=0.5]{img/penrose-triangle.eps}
 \captionsetup{labelformat=empty}
 \caption{彭罗斯三角形}
 \label{fig:Penrose-triangle}
\end{wrapfigure}

记得在中学数学课上,老师会在黑板上写一个充满字母的式子,然后让同学们化简。有同学会自告奋勇到讲台上拿起粉笔在黑板上推导。合并同类项、因式分解、各种办法都可以用。这个过程就像是变魔术,最后往往得到意想不到的简单结果。当然也有卡住或者绕圈子的时候,老师总是耐心地提示,引导我们找到思路。

这样的经历好像就发生在眼前,一方面,满手的粉笔灰让同学们体会到老师的不易,另一方面,那种推理的神秘强力量让我感到它的强大。我总是希望知道更多的公式,这样就能在化简或者推导时派上用场。

这种推理的神奇之处在于,我们不用特别关心这些公式或者定理在当时场景下的具体含义。就像摆动积木一样,从散落的各种零件,最后搭建起一个有趣玩具。这些公式和定理相互组装到一起,最后引向一个有趣的结果。看到$a^2 + 2ab + b^2$就会把它变换为$(a+b)^2$,就像把两块积木插到一起那样自然,我们不用在推理时强迫自己回想这个公式的几何意义。

\begin{figure}[htbp]
\centering
\begin{tikzpicture}
\draw[fill=gray, draw=black, pattern=north west lines]
  (0, 0) rectangle (4, 4);
\filldraw[fill=white]
  (0, 0) rectangle (1, 1)
  (1, 1) rectangle (4, 4);
\path (0.5, 0.5) node {$a^2$}
      (2.5, 2.5) node {$b^2$}
      (-0.5, 0.5) node {$a$}
      (2.5, 0.5) node {$ab$}
      (4.5, 2.5) node {$b$}
      (0.5, 2.5) node {$ab$};
\path (0, -1) node (l) {}
      (2, -1) node (m) {$a + b$}
      (4, -1) node (r) {};
\draw[->] (m) -- (l);
\draw[->] (m) -- (r);
\end{tikzpicture}
\caption{$(a + b)^2 = a^2 + 2ab + b^2$的几何意义}
\end{figure}

本章中,我们用两个例子说明如何进行编程中的推理。每个例子都首先用直观的方法给出分析和解释,然后再用纯推理的方式给出另一个解法。这就像$(a+b)^2$的情形。一方面我们可以用几何的直观,将其理解为一大一小两个小正方形和两个相等的矩形的面积;另一方面,我们也可以用纯推理一步一步导出同样的结果。

\[
\begin{array}{rcll}
(a + b)^2 & = & (a + b)(a + b) & \text{二次方的定义} \\
          & = & a(a + b) + b(a + b) & \text{乘法分配律} \\
          & = & a^2 + ab + ba + b^2 & \text{再次用分配律} \\
          & = & a^2 + 2ab + b^2 & \text{合并同类项$ab$和$ba$}
\end{array}
\]

\section{构建——叠加的融合}

我们要举的第一个例子是构建——叠加的融合(build/foldr fusion law)。2015年Java在其1.8版本中加入了lambda表达式并且提供了一系列支持函数式编程的工具。但是有人很快发现,尽管一连串地函数调用表达能力很强,简洁优雅,但是性能会下降很多。原因之一就是这些串起来函数调用产生了大量中间结果。这些中间结果往往不是一两个简单的数值,而通常是列表、容器这样规模很大的结构。这些结构被下一个函数消费使用,然后就丢弃了。但是接下来会产生另一个同等规模的结构。这种产生——一次性消费——丢弃——再产生的过程,沿着函数调用链一环一环地重复,造成了很大浪费。

例如\cite{GLPJ-1993},我们想判断一个列表中的每个元素是否都满足某个条件。可以这样定义:

\[
all(p, xs) = and(map(p, xs))
\]

传入$all(prime, [2, 3, 5, 7, 11, 13, 17, 19, ...])$就可以判断是否列表中都是素数。但是这个实现的效率却不高。首先$map(prime, xs)$会产生一个和$xs$同样长度的列表,列表中的每个元素是一个布尔值[True, True, ...],每个布尔值表示对应的元素是不是素数。然后这个布尔值列表传入$and$函数,检查是否存在False。最后$xs$和布尔值列表都被丢弃,而仅仅返回一个布尔值作为最终结果。

下面是另一种定义,它能够避免产生中间的布尔值列表:

\[
\begin{array}{l}
all(p, xs) = h(xs) \\
  \begin{cases}
  h([]) = True \\
  h(x:xs) = p(x) \land h(xs) \\
  \end{cases}
\end{array}
\]

虽然这个实现不产生中间结果,可是和前面的$and(map(p, xs)$比起来,既冗长又不直观。有没有什么办法,鱼和熊掌兼得,即不丧失直观性,又能避免低效的实现呢?我们发现有些变换满足这一要求。例如:

\[
map\ sqrt\  (map\ abs\ xs) = map\ (sqrt \circ abs) xs
\]

先把列表中每个元素取绝对值构成一列新数,然后再把这列数中的每个开方。这和把列表中每个数先取绝对值然后再立即开方后构成一列新数等价。由此我们可以得到一个转换规则:

\be
map\ f\ (map\ g\ xs) = map\ (f \circ g)\ xs
\ee

但是这样的规则太多了,我们无法全部把它们列出。并且在千变万化的程序中,我们无法一眼就看出应该用哪一条规则优化。吉尔(Gill)、朗奇布瑞(Launchbury)、佩顿琼斯(Peyton Jones)在1993年提出了一个方法,他们从列表最本质的构造和叠加操作入手,找到了优化的规律。

我们在第一章就给出过列表的叠加操作,它的定义为:

\[
\begin{array}{l}
foldr\ \oplus\ z\ [] = z \\
foldr\ \oplus\ z\ (x:xs) = x\ \oplus\ (foldr\ \oplus\ z\ xs) \\
\end{array}
\]

\section{KMP匹配算法}

我们给出的第二个例子是著名的KMP字符串匹配算法。

\cite{Bird-2010} pp. 117-135

\ifx\wholebook\relax \else
\begin{thebibliography}{99}

\bibitem{GLPJ-1993}
Andrew Gill, John Launchbury, Simon L. Peyton Jones. ``A Short Cut to Deforestation''. Functional programming languages and computer architecture. pp. 223-232. 1993.

\bibitem{Bird-2010}
Richard Bird. ``Pearls of Functional Algorithm Design''. Cambridge University Press; 1 edition November 1, 2010. ISBN: 978-0521513388.

\end{thebibliography}

\expandafter\enddocument
%\end{document}

\fi
