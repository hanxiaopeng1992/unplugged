\ifx\wholebook\relax \else

\documentclass{article}

\input{../common-zh-cn.tex}

\setcounter{page}{1}

\begin{document}

\title{推理}

\author{刘新宇
\thanks{{\bfseries 刘新宇} \newline
  Email: liuxinyu95@gmail.com \newline}
  }

\maketitle
\fi

\markboth{推理}{编程的数学原理}

\ifx\wholebook\relax
\chapter{推理}
\numberwithin{Exercise}{chapter}
\fi

\epigraph{3表示2+1,4表示3+1。所以接下来(虽然证明很长),4等于2+2。因此, 数学知识不再是神秘的。}{——罗素}

% ... mathematical knowledge ... is, in fact, merely verbal knowledge. "3" means "2+1", and "4" means "3+1". Hence it follows (though the proof is long) that "4" means the same as "2+2". Thus mathematical knowledge ceases to be mysterious.  -- Bertrand Russell

\begin{wrapfigure}{R}{0.4\textwidth}
 \centering
 \includegraphics[scale=0.5]{img/penrose-triangle.eps}
 \captionsetup{labelformat=empty}
 \caption{彭罗斯三角形}
 \label{fig:Penrose-triangle}
\end{wrapfigure}

记得在中学数学课上,老师会在黑板上写一个充满字母的式子,然后让同学们化简。有同学会自告奋勇到讲台上拿起粉笔在黑板上推导。合并同类项、因式分解、各种办法都可以用。这个过程就像是变魔术,最后往往得到意想不到的简单结果。当然也有卡住或者绕圈子的时候,老师总是耐心地提示,引导我们找到思路。

这样的经历好像就发生在眼前,一方面,满手的粉笔灰让同学们体会到老师的不易,另一方面,那种推理的神秘强力量让我感到它的强大。我总是希望知道更多的公式,这样就能在化简或者推导时派上用场。

这种推理的神奇之处在于,我们不用特别关心这些公式或者定理在当时场景下的具体含义。就像摆动积木一样,从散落的各种零件,最后搭建起一个有趣玩具。这些公式和定理相互组装到一起,最后引向一个有趣的结果。看到$a^2 + 2ab + b^2$就会把它变换为$(a+b)^2$,就像把两块积木插到一起那样自然,我们不用在推理时强迫自己回想这个公式的几何意义。

\begin{figure}[htbp]
\centering
\begin{tikzpicture}
\draw[fill=gray, draw=black, pattern=north west lines]
  (0, 0) rectangle (4, 4);
\filldraw[fill=white]
  (0, 0) rectangle (1, 1)
  (1, 1) rectangle (4, 4);
\path (0.5, 0.5) node {$a^2$}
      (2.5, 2.5) node {$b^2$}
      (-0.5, 0.5) node {$a$}
      (2.5, 0.5) node {$ab$}
      (4.5, 2.5) node {$b$}
      (0.5, 2.5) node {$ab$};
\path (0, -1) node (l) {}
      (2, -1) node (m) {$a + b$}
      (4, -1) node (r) {};
\draw[->] (m) -- (l);
\draw[->] (m) -- (r);
\end{tikzpicture}
\caption{$(a + b)^2 = a^2 + 2ab + b^2$的几何意义}
\end{figure}

\ifx\wholebook\relax \else
\begin{thebibliography}{99}

\bibitem{Dieudonne1987}
[法]让$\cdot$迪厄多内 著,沈用欢 译 ``当代数学,为了人类心智的荣耀''. 上海教育出版社. 2000年3月. ISBN: 7532063062

\end{thebibliography}

\expandafter\enddocument
%\end{document}

\fi
