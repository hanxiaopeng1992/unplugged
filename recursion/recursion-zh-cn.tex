\ifx\wholebook\relax \else

\documentclass[UTF8]{article}

\input{../common-zh-cn.tex}

\setcounter{page}{1}

\begin{document}

\title{递归}

\author{刘新宇
\thanks{{\bfseries 刘新宇} \newline
  Email: liuxinyu95@gmail.com \newline}
  }

\maketitle
\fi

\markboth{递归}{编程的数学原理}

\epigraph{\textbf{造物神}是\textbf{造}物主\textbf{物}色的\textbf{神}怪}{——[美]侯世达《哥德尔、埃舍尔、巴赫——集异壁之大成》}

\section{万物皆数}

\begin{wrapfigure}{R}{0.3\textwidth}
%\begin{figure}[htbp]
 \centering
 \includegraphics[scale=0.4]{img/Pythagoras.eps}
 \captionsetup{labelformat=empty}
 \caption{毕达哥拉斯}
 \label{fig:Pythagoras}
%\end{figure}
\end{wrapfigure}

毕达哥拉斯的介绍,万物皆数的哲学思想(三角形数,音乐)。公度的概念,无理数的发现和递归的关系。


%\begin{wrapfigure}{R}{0.5\textwidth}
%% \begin{figure}[htbp]
%%  \centering
%%  \includegraphics[scale=0.6]{img/Babylonian_numerals.eps}
%%  \caption{巴比伦楔形文字中的数字\cite{wiki-babylonian-num}。}
%%  \label{fig:babylonian-num}
%% \end{figure}
%\end{wrapfigure}

%\begin{wrapfigure}{R}{0.5\textwidth}
%% \begin{figure}[htbp]
%%  \centering
%%  \includegraphics[scale=0.2]{img/abstract-num.eps}
%%  \caption{具体的三个事物和抽象的数字三}
%%  \label{fig:abstract-num}
%% \end{figure}
%\end{wrapfigure}

\section{欧几里得算法}

%\begin{wrapfigure}{R}{0.4\textwidth}
%% \begin{figure}[htbp]
%%  \centering
%%  \includegraphics[scale=0.2]{img/Peano.eps}
%%  \caption{朱塞佩$\cdot$皮亚诺(Giuseppe Peano)1858 - 1932。}
%%  \label{fig:Peano}
%% \end{figure}
%\end{wrapfigure}

欧几里得最大公约数(公度)算法的概念。几何量和数的分离。最大公约数算法的意义,递归原理问题的提出

扩展欧几里得算法、倒水趣题。

\section{$\lambda$演算}

lambda的由来,三大变换:Alpha, Beta, Eta变换,变换与归约。

%\begin{figure}[htbp]
\begin{wrapfigure}{R}{0.4\textwidth}
\centering
\begin{tikzpicture}[scale=0.8]
\filldraw[fill=gray, draw=black, pattern=north west lines] (0, 0) rectangle (2, 1)
    (2, 0) rectangle (3, 1);
\draw (3, 0) rectangle (4.5, 1);
\draw (0, -1) rectangle (2, -2);
\filldraw[fill=gray, draw=black, pattern=north west lines] (2, -1) rectangle (3, -2)
    (3, -1) rectangle (4.5, -2);
\end{tikzpicture}
\caption{加法结合律的几何证明。上下面积相等}
\end{wrapfigure}
%\end{figure}

%\begin{wrapfigure}{R}{0.4\textwidth}
\begin{figure}[htbp]
\centering
\begin{tikzpicture}[scale=0.8]
\draw (0, 0) rectangle (2, 1)
    (2, 0) rectangle (3, 1);
\draw (0, -1) rectangle (1, -2)
    (1, -1) rectangle (3, -2);
\end{tikzpicture}
\caption{加法交换律的几何证明。将上方的图形倒过来看,或者在镜中看。}
\end{figure}
%\end{wrapfigure}

\begin{Exercise}
\begin{enumerate}
\item 一些lambda演算的练习
\end{enumerate}
\end{Exercise}

\section{递归的定义}

Y组合子和递归的定义
%% \begin{wrapfigure}{R}{0.4\textwidth}
%% %\begin{figure}[htbp]
%%  \centering
%%  \includegraphics[scale=0.4]{img/Fibonacci.eps}
%%  \caption{比萨的列奥纳多,又称斐波那契(Leonardo Pisano, Fibonacci),1175年-1250年}
%%  \label{fig:Fibonacci}
%% %\end{figure}
%% \end{wrapfigure}


%\begin{wrapfigure}{L}{0.3\textwidth}
%% \begin{figure}[htbp]
%%  \centering
%%  \includegraphics[scale=0.5]{img/fibonacci_spiral.eps}
%%  \caption{这些正方形的边长组成了斐波那契序列。}
%%  \label{fig:fibonacci_spiral}
%% \end{figure}
%\end{wrapfigure}

%\begin{wrapfigure}{R}{0.3\textwidth}
%% \begin{figure}[htbp]
%%  \centering
%%  \includegraphics[scale=0.2]{img/PWW.eps}
%%  \caption{《无需语言的证明》封面局部}
%%  \label{fig:PWW}
%% \end{figure}
%\end{wrapfigure}

\section{$\lambda$的意义}

CONS/HEAD/TAIL的lambda表示。

二叉树、多叉树的递归结构与FOLD

\section{递归的结构与形式}

%\begin{wrapfigure}{R}{0.3\textwidth}
%% \begin{figure}[htbp]
%%  \centering
%%  \includegraphics[scale=1.0]{img/the-school-of-athens.eps}
%%  \caption{拉斐尔《雅典学院》局部}
%%  \label{fig:the-school-of-athens}
%% \end{figure}
%\end{wrapfigure}

递归的形式之美——分形与艺术

递归的结构之美——文艺和音乐作品

\ifx\wholebook\relax \else
\begin{thebibliography}{99}

\bibitem{wiki-number}
Wikipedia. ``古代计数系统的历史''. \url{https://en.wikipedia.org/wiki/History_of_ancient_numeral_systems}

\bibitem{trip-to-number-kindom}
[美]卡尔文$\cdot$C$\cdot$克劳森. ``数学旅行家:漫游数王国''. 袁向东、袁钧译,上海教育出版社。ISBN: 7-5320-7883-3/G $cdot$ 7972

\bibitem{wiki-babylonian-num}
Wikipedia. ``古巴比伦数字''. \url{https://en.wikipedia.org/wiki/Babylonian_numerals}

\bibitem{GEB}
[美]候世达 ``哥德尔、埃舍尔、巴赫——集异壁之大成''. 商务印书馆 1996. ISBN: 978-7-100-01323-9

\bibitem{Bird97}
Richard Bird, Oege de Moor. ``Algebra of Programming''. University of Oxford, Prentice Hall Europe. 1997. ISBN: 0-13-507245-X.

\bibitem{Gusen2014}
顾森 ``浴缸里的惊叹''. 人民邮电出版社. 2014, ISBN: 9787115355744

\end{thebibliography}

\expandafter\enddocument
%\end{document}

\fi
