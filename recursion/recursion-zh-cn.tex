\ifx\wholebook\relax \else

\documentclass[UTF8]{article}

\input{../common-zh-cn.tex}

\setcounter{page}{1}

\begin{document}

\title{递归}

\author{刘新宇
\thanks{{\bfseries 刘新宇} \newline
  Email: liuxinyu95@gmail.com \newline}
  }

\maketitle
\fi

\markboth{递归}{编程的数学原理}

\epigraph{\textbf{造物神}是\textbf{造}物主\textbf{物}色的\textbf{神}怪}{——[美]侯世达《哥德尔、埃舍尔、巴赫——集异壁之大成》}

\begin{wrapfigure}{R}{0.3\textwidth}
%\begin{figure}[htbp]
 \centering
 \includegraphics[scale=0.4]{img/Pythagoras.eps}
 \captionsetup{labelformat=empty}
 \caption{毕达哥拉斯(约前570——前490)}
 \label{fig:Pythagoras}
%\end{figure}
\end{wrapfigure}

人们通过深入了解数进而了解自然。在上一章中,我们介绍了自然数的皮亚诺公理。并且展示了一些和自然数有着相同结构的事物,包括计算机系统中的基本数据结构列表。自然数成为了我们进一步前进的基石。但是我们的大厦还不稳固。第一章中,我们不加证明地使用了递归的概念。例如阶乘的定义:

\[
\begin{array}{l}
fact(0) = 1 \\
fact(n + 1) = (n + 1) \cdot fact(n)
\end{array}
\]

以及$foldn$的实现:

\[
\begin{array}{l}
foldn(z, f, 0) = z \\
foldn(z, f, n') = f(foldn(z, f, n))
\end{array}
\label{eq:foldn}
\]

递归的原理是什么?为什么它是正确的?递归可以在更低的层次被表示么?这些都是我们在这一章要解决的问题。

\section{万物皆数}

从数出发研究世间万物的第一人要算是古希腊的数学家和哲学家毕达哥拉斯了。他的名字通过著名的勾股定理(在西方叫毕达哥拉斯定理)而家喻户晓。毕达哥拉斯出生于希腊的萨摩斯(Samos)岛,年轻时他曾去米利都(Miletus)向古希腊哲学的奠基人泰勒斯(Thales)学习。在泰勒斯的建议下,毕达哥拉斯前往东方学习数学。他在埃及学习了13年(一说为22年)。后来波斯帝国征服了埃及,他又随军向东到达了巴比伦,向巴比伦人学习数学和天文知识。或许后来他还到达了更远的印度。不论到了哪里,毕达哥拉斯都不断向有学问的人请教,丰富自己的见解。重要的是,他不仅刻苦学习,而且更善于思考。在经过兼收并蓄、汲取各家之长后,毕达哥拉斯形成并完善了自己的思想\cite{HanXueTao16}。

经历了漫长的在外游历后,这位年近半百的智者返回了故乡并开始讲学。公元前520年左右,为了摆脱当地的暴政,毕达哥拉斯移居到了意大利南部的克罗顿(Croton)发展,在那里他赢得了人们的信任与景仰。毕达哥拉斯的弟子中还有女性,他们把主要的精力都用来研究天文、几何、数论及音乐这四门学科。它们被称为四术(quadrivium),影响了欧洲教育两千多年\cite{StepanovRose15}。四术体现了毕达哥拉斯“万物皆数”的哲学思想。星体的运动与几何对应,而几何又以数为基础,数字还可以衍生出音乐。毕达哥拉斯是首个发现纯八度音(octave)在频率上有数学规律的人。他的弟子说他可以“听见天界的乐音”。


毕达哥拉斯的介绍,万物皆数的哲学思想(三角形数,音乐)。公度的概念,无理数的发现和递归的关系。


%\begin{wrapfigure}{R}{0.5\textwidth}
%% \begin{figure}[htbp]
%%  \centering
%%  \includegraphics[scale=0.6]{img/Babylonian_numerals.eps}
%%  \caption{巴比伦楔形文字中的数字\cite{wiki-babylonian-num}。}
%%  \label{fig:babylonian-num}
%% \end{figure}
%\end{wrapfigure}

%\begin{wrapfigure}{R}{0.5\textwidth}
%% \begin{figure}[htbp]
%%  \centering
%%  \includegraphics[scale=0.2]{img/abstract-num.eps}
%%  \caption{具体的三个事物和抽象的数字三}
%%  \label{fig:abstract-num}
%% \end{figure}
%\end{wrapfigure}

\section{欧几里得算法}

%\begin{wrapfigure}{R}{0.4\textwidth}
%% \begin{figure}[htbp]
%%  \centering
%%  \includegraphics[scale=0.2]{img/Peano.eps}
%%  \caption{朱塞佩$\cdot$皮亚诺(Giuseppe Peano)1858 - 1932。}
%%  \label{fig:Peano}
%% \end{figure}
%\end{wrapfigure}

欧几里得最大公约数(公度)算法的概念。几何量和数的分离。最大公约数算法的意义,递归原理问题的提出

扩展欧几里得算法、倒水趣题。

\section{$\lambda$演算}

lambda的由来,三大变换:Alpha, Beta, Eta变换,变换与归约。

%\begin{figure}[htbp]
\begin{wrapfigure}{R}{0.4\textwidth}
\centering
\begin{tikzpicture}[scale=0.8]
\filldraw[fill=gray, draw=black, pattern=north west lines] (0, 0) rectangle (2, 1)
    (2, 0) rectangle (3, 1);
\draw (3, 0) rectangle (4.5, 1);
\draw (0, -1) rectangle (2, -2);
\filldraw[fill=gray, draw=black, pattern=north west lines] (2, -1) rectangle (3, -2)
    (3, -1) rectangle (4.5, -2);
\end{tikzpicture}
\caption{加法结合律的几何证明。上下面积相等}
\end{wrapfigure}
%\end{figure}

%\begin{wrapfigure}{R}{0.4\textwidth}
\begin{figure}[htbp]
\centering
\begin{tikzpicture}[scale=0.8]
\draw (0, 0) rectangle (2, 1)
    (2, 0) rectangle (3, 1);
\draw (0, -1) rectangle (1, -2)
    (1, -1) rectangle (3, -2);
\end{tikzpicture}
\caption{加法交换律的几何证明。将上方的图形倒过来看,或者在镜中看。}
\end{figure}
%\end{wrapfigure}

\begin{Exercise}
\begin{enumerate}
\item 一些lambda演算的练习
\end{enumerate}
\end{Exercise}

\section{递归的定义}

Y组合子和递归的定义
%% \begin{wrapfigure}{R}{0.4\textwidth}
%% %\begin{figure}[htbp]
%%  \centering
%%  \includegraphics[scale=0.4]{img/Fibonacci.eps}
%%  \caption{比萨的列奥纳多,又称斐波那契(Leonardo Pisano, Fibonacci),1175年-1250年}
%%  \label{fig:Fibonacci}
%% %\end{figure}
%% \end{wrapfigure}


%\begin{wrapfigure}{L}{0.3\textwidth}
%% \begin{figure}[htbp]
%%  \centering
%%  \includegraphics[scale=0.5]{img/fibonacci_spiral.eps}
%%  \caption{这些正方形的边长组成了斐波那契序列。}
%%  \label{fig:fibonacci_spiral}
%% \end{figure}
%\end{wrapfigure}

%\begin{wrapfigure}{R}{0.3\textwidth}
%% \begin{figure}[htbp]
%%  \centering
%%  \includegraphics[scale=0.2]{img/PWW.eps}
%%  \caption{《无需语言的证明》封面局部}
%%  \label{fig:PWW}
%% \end{figure}
%\end{wrapfigure}

\section{$\lambda$的意义}

CONS/HEAD/TAIL的lambda表示。

二叉树、多叉树的递归结构与FOLD

\section{递归的结构与形式}

%\begin{wrapfigure}{R}{0.3\textwidth}
%% \begin{figure}[htbp]
%%  \centering
%%  \includegraphics[scale=1.0]{img/the-school-of-athens.eps}
%%  \caption{拉斐尔《雅典学院》局部}
%%  \label{fig:the-school-of-athens}
%% \end{figure}
%\end{wrapfigure}

递归的形式之美——分形与艺术

递归的结构之美——文艺和音乐作品

\ifx\wholebook\relax \else
\begin{thebibliography}{99}

\bibitem{HanXueTao16}
韩雪涛 ``数学悖论与三次数学危机''. 人民邮电出版社. 2016, ISBN: 9787115430434

\bibitem{StepanovRose15}
[美] 亚历山大 A$cdot$斯捷潘诺夫,丹尼尔 E$\cdot$罗斯著,爱飞翔译. ``数学与泛型编程:高效编程的奥秘''. 机械工业出版社. 2017, ISBN: 9787111576587

\end{thebibliography}

\expandafter\enddocument
%\end{document}

\fi
