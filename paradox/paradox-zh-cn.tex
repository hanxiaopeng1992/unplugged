\ifx\wholebook\relax \else

\documentclass{article}

\input{../common-zh-cn.tex}

\setcounter{page}{1}

\begin{document}

\title{悖论}

\author{刘新宇
\thanks{{\bfseries 刘新宇} \newline
  Email: liuxinyu95@gmail.com \newline}
  }

\maketitle
\fi

\markboth{悖论}{编程的数学原理}

\ifx\wholebook\relax
\chapter{悖论}
\numberwithin{Exercise}{chapter}
\fi

\epigraph{除了自己的无知,我什么都不懂。}{——苏格拉底}

\begin{wrapfigure}{R}{0.5\textwidth}
 \centering
 \includegraphics[scale=0.3]{img/Escher-Waterfall-1961.eps}
 \captionsetup{labelformat=empty}
 \caption{埃舍尔《瀑布》1961}
 \label{fig:Escher-Waterfall}
\end{wrapfigure}

回顾人类面临人工智能挑战的情况,国际象棋、围棋、绘画、医疗影像判断等。
提出计算机能否战胜人类的问题

\section{计算的边界}

图灵停机问题

停机问题类比,并引出罗素悖论

\section{罗素悖论}

\section{数学基础的分歧}

逻辑主义

直觉主义

形式主义

\section{公理集合论}

\section{哥德尔不完全性定理}

\section{不完全性定理的证明}

\section{万能的程序与对角线证明}

\section{尾声}
理性思维的边界。

埃舍尔的龙

\ifx\wholebook\relax \else
\begin{thebibliography}{99}

\bibitem{De-linfini-2018}
[法] 让-皮埃尔$\cdot$卢米涅,马克$\cdot$拉雪茨-雷 著,孙展 译. 从无穷开始——科学的困惑与疆界. 人民邮电出版社. 2018. ISBN: 9787115479198

\bibitem{Noguchi2007}
[日] 野口哲也 著,刘慧 韩丽红 译. 数学原来可以这样学. 湖南人民出版社. 2014. ISBN: 9787556100897
% Tetsunori Noguchi. SUGAKUTEKI SENSE GA MINITUKU RENSHUCHO.

\bibitem{Wikipedia-Googol}
Wikipedia. ``Googol''. \url{https://en.wikipedia.org/wiki/Googol}

\bibitem{Wikipedia-Zeno}
Wikipedia. ``Zeno's Paradoxes''. \url{https://en.wikipedia.org/wiki/Zeno's_paradoxes}

\bibitem{HanXueTao16}
韩雪涛 ``数学悖论与三次数学危机''. 人民邮电出版社. 2016, ISBN: 9787115430434

\bibitem{Elements}
[古希腊] 欧几里得 著,兰纪正 朱恩宽 译,梁宗巨 张毓新 徐伯谦 校订 ``几何原本''. 译林出版社. 2014, ISBN: 9787544750066

\bibitem{M-Kline-2007}
[美] M$\cdot$克莱因 著 李宏魁 译 ``数学:确定性的丧失'' 湖南科学技术出版社,2007年4月 ISBN: 978-7-5357-1857-0
% Morris Kline ``Mathematics: The Loss of Certainty''. Oxford University Press, 1980.

\bibitem{Stepanov}
Stepanov and Rose. ``数学与泛型编程''. 爱飞翔译,机械工业出版社。ISBN: 978-7-111-57658-7. 2017.

\bibitem{GEB}
[美]候世达 ``哥德尔、埃舍尔、巴赫——集异壁之大成''. 商务印书馆 1996. ISBN: 978-7-100-01323-9

\bibitem{GCH}
张锦文,王雪生 ``连续统假设''. 世界数学名题欣赏丛书。辽宁教育出版社 1988. ISBN: 7-5382-0436-9/G$\cdot$445

\end{thebibliography}

\expandafter\enddocument
%\end{document}

\fi
