\ifx\wholebook\relax \else

\documentclass{article}

\input{../common-zh-cn.tex}

\setcounter{page}{1}

\begin{document}

\title{悖论}

\author{刘新宇
\thanks{{\bfseries 刘新宇} \newline
  Email: liuxinyu95@gmail.com \newline}
  }

\maketitle
\fi

\markboth{悖论}{编程的数学原理}

\ifx\wholebook\relax
\chapter{悖论}
\numberwithin{Exercise}{chapter}
\fi

\epigraph{除了自己的无知,我什么都不懂。}{——苏格拉底}

\begin{wrapfigure}{R}{0.5\textwidth}
 \centering
 \includegraphics[scale=0.3]{img/Escher-Waterfall-1961.eps}
 \captionsetup{labelformat=empty}
 \caption{埃舍尔《瀑布》1961}
 \label{fig:Escher-Waterfall}
\end{wrapfigure}

1996年,第26界国际奥林匹克运动会正在美国的亚特兰大举行。来自世界各地的选手在速度、力量、技巧上展开竞赛挑战人类的极限。与此同时,还进行着另一场有趣的竞赛。超级计算机“深蓝”与人类的国际象棋世界冠军卡斯帕罗夫展开了对抗赛。比赛结果是深蓝2胜4负输给了人类象棋冠军。翌年,改进的深蓝再次向卡斯帕罗夫发起挑战。5月11日,计算机在正常时限的比赛中首次击败了卡斯帕罗夫。总比分1胜2负3平。深蓝计算机重1270公斤,有32个微处理器,每秒钟可以计算2亿步。为了能让"深蓝”挑战人类冠军,设计小组输入了一百多年来优秀棋手的对局两百多万对局。人类用智慧创造的机器,在人类骄傲的智慧领域首次击败了人类——这一结局引发了关注、恐惧、和激烈的讨论。

在当时,人们普遍认为,这是人工智能的一大进步。尽管在国际象棋上取得了巨大进步,但是在围棋上,计算机和人类仍存在巨大差距。对于国际象棋来说,棋盘8行8列,32枚棋子。计算机要在$10^{123}$这样巨大的博弈树中进行搜索。即使深蓝美妙能算2亿步,遍历博弈树仍需要近$10^{107}$年。为此深蓝的设计小组通过计算机程序缩小了搜索空间,使得深蓝能够搜索当前棋局后面的12步棋。而一般好的人类棋手,大约只能估计到10步左右。但是围棋的棋盘有19行19列,在一共361个格点上可以放置黑色或者白色的棋子。博弈树的规模为$10^{360}$,远远超越国际象棋。所以在之后的一段时间里,人们仍然不相信计算机可以挑战我们。

%\begin{wrapfigure}{R}{0.5\textwidth}
\begin{figure}[htbp]
 \centering
 \includegraphics[scale=0.4]{img/Deep-blue-1997.eps}
 \captionsetup{labelformat=empty}
 \caption{卡斯帕罗夫在与深蓝对弈,图片原载《科学美国人杂志》}
 \label{fig:Deep-blue-1997}
\end{figure}
%\end{wrapfigure}

时光匆匆过去了20年,2016年,计算机程序Alpha-Go向人类的围棋大师展开了挑战。韩国的九段棋手李世石以1比4的总比分输掉了比赛。一年后,Alpha-Go再次以3局全胜的成绩战胜了中国棋手柯洁。被人们认为是人工智能游戏“圣杯”的围棋终于被攻破了。面对没有感情的计算机,柯洁心有不甘,潸然落泪。作为人类,我们的心情很复杂。即使是从事智力工作的程序员群体也感到了来自机器的压力——我们是否会被机器取代?

%University of Tubingen, Germany
% Leon A. Gatys, Alexander S. Ecker Matthias Bethge
传统上我们认为,艺术文学等领域,涉及人们的文化背景、内在感情和与生俱来的性格因素,是无法被机器所替代的。2015年,德国斯图加特以南40公里的小镇图宾根大学大学的盖提斯、埃克、贝特格三位研究人员利用机器学习人类艺术家的风格,把图宾根镇的一张风景照片变换成了不同风格的艺术画作\cite{Gatys-2015}。无论是后印象派大师梵高色彩强烈夸张的画风,还是透纳那浪漫主义水天浑浊的光影效果,都被机器模仿得惟妙惟肖。犹如大师本人所作(图\ref{fig:style-transfer})。

%\begin{wrapfigure}{R}{0.5\textwidth}
\begin{figure}[htbp]
 \centering
 \includegraphics[scale=0.85]{img/style-transfer.eps}
 %\captionsetup{labelformat=empty}
 \caption{机器学习产生的不同艺术风格的画作:A,图宾根镇的风景照片;B,英国画家透纳1810年的原作《运输船遇难》和透纳风格的画作;C,荷兰后印象派画家梵高1889年的原作《星空》和梵高风格的画作;D,挪威表现主义画家爱德华$\cdot$蒙克1893年的原作《呐喊》和蒙克风格的画作;E,西班牙现代艺术家毕加索1910年的原作《坐着的裸女》和毕加索风格的画作;F,俄罗斯抽象艺术先驱画家康定斯基1913年的原作《构成第七号》和康定斯基风格的画作}
 \label{fig:style-transfer}
\end{figure}
%\end{wrapfigure}

在随后的数年中,人工智能和机器学习突飞猛进地进入了各种领域。机器产生不同音乐家风格的音乐,能够演奏出紧张、舒缓等不同的情绪的旋律和节奏,而不再是呆板单调的电子琴音。机器批量翻译新闻稿和各种学术论文,和专业翻译的文笔不相上下。机器处理X片、CT、核磁共振等医学图像并给出病理诊断,并且结果在准确程度上超越人类医生,人工智能操控的无人车在街道上行驶,成功超过其它车辆并避让行人。无人值守的商店突然出现在街边,人们可以直接从货架上拿走商品,并在走出商店的一刻自动支付……作为人类的我们不禁会问:我们消灭工作岗位的速度是否会超过创造工作机会的速度?人类是否会被机器全面取代?机器是否最终会统治我们?

所有这些在本质上都可以归结到一个问题:计算的能力是否存在边界?如果有的话,计算的边界在哪里?

\section{计算的边界}

顾森在《思考的乐趣》一书中讲到,注视着一个运行了很久的程序时的两难心情:这个程序能结束么?是应该继续等下去,还是杀掉进程强行结束?有没有什么编译器能事先告诉你的程序是否会无限运行下去?(\cite{GuSen-2012},第228页)

\begin{quotation}
为什么不可能呢?这个东西看上去比时光旅行机更现实一些。或许我们会在某个科幻电影中看到,一个程序员在漆黑的屏幕上输入几个数,敲了一下回车,然后屏幕上立即用高亮加粗字体显示:“警告:该输入数据会导致程序无限运行下去,确定执行?(Y/N)”如果有一天,这一切真的成为了现实。那么你能利用这个玩意儿来做些什么使用、有价值的事情?如果我说你能靠这玩意儿发大财的话,你相信么?……我上来就先写一个哥德巴赫猜想的验证程序。我写一个程序,让他从小到大枚举所有的偶数,看是不是有两个质数加起来等于它。如果找到来,继续枚举下一个偶数,否则输出反例并结束程序。然后编译该程序。这个编译器不是可以预先判断我这个程序能否终止吗?如果编译器说我这个程序会无限执行下去的话,我岂不是相当于证实来哥德巴赫猜想吗?或者,编译器说程序会最终终止,那哥德巴赫猜想不就直接被推翻来吗?不管怎样,我都将成为解决哥德巴赫猜想的第一人,在数学史上留下自己的名字。接下来呢?把刚才的程序代码改成孪生素数搜索器,在利用编译器检查一下,看看是不是真的有无穷多个孪生素数。梅森素数是否有无穷多个,这个也是数论中长期以来悬而未决的难题。不过现在看来,我也能不费吹灰之力就把它解决了。还记得$3x+1$问题吗?写一个“证明程序”也只是几分钟的事情,而且还能拿走埃尔德什提供的500美元奖金呢。数学上的未解之谜多着呢。我永远不愁没事做。1984年,马丁$\cdot$拉巴尔询问能否用9个不同的平方数构成$3 \times 3$幻方,这个问题的奖金目前已经积累到了100美元加100欧元再加一瓶香槟。网上搜索“数学未解难题”,看看哪些问题是离散的,其中又有哪些问题是有悬赏的,写几个程序就可以把它们统统解决……
\end{quotation}

1936年,计算机科学和人工智能的先驱图灵证明了一个命题:不存在可以判断任何程序是否可以停机的通用算法。证明的核心部分包含了计算机程序的数学定义——图灵机模型。后人称这一问题为图灵停机问题。

为了图灵证明停机问题,我们采用反证法,假设存在一个名叫$halts(p)$的算法,能够判断任意程序$p$是否停机。首先我们定义一个永不停机的程序:

\[
forever() = forever()
\]

这是一个无穷递归的调用。然后我们构造一个名为$G$的特殊程序\footnote{我们用字母$G$是有特殊用意的,$G$是哥德尔的首字母,它恰好和哥德尔不完全定理中不可判定命题的名字相同。},它的定义如下:

\[
G() = \begin{cases}
halts(G) = \text{停机}: & forever() \\
\text{否则}: & \text{停机} \\
\end{cases}
\]

在程序$G$中,我们通过$halts(G)$判断$G$本身是否停机。如果停机,我们就调用$forever()$永远运行下去。但这恰恰说明$G$不会停机,所以$halts(G)$应该为假,但是按照上面定义的第二行,此时我们停机。这恰恰说明$halts(G)$应该为真。所以不论$halts(G)$是真是假,我们都会得到矛盾的结论。因此我们最初的假设不成立,也就是说,不存在一个可以判断任意程序能否停机的通用算法。

也有一种分两步证明图灵停机问题的方法(\cite{SICP},第268页),前面都一样,但在构造$G$时,$G$接受一个参数$p$,它把$p$应用到自身上并传给$halts$:

\lstset{frame=single}
\begin{lstlisting}
G(p) = if halts(p(p)) then forever() else 'Halted'
\end{lstlisting}

接下来的一步中,我们把$G$传给自己$G(G)$看发生了什么?此时如果$halts(G(G))$返回真,则接下来运行$forever()$,所以$G(G)$永远运行不会停机。但这恰恰说明$halts(G(G))$应该返回假,所以接下来程序进入\texttt{else}分支,返回停机。但这又说明$halts(G(G))$应该返回真。所以不管停机与否,都陷入了矛盾之中。

伟大的图灵停机定理清晰地给出了一个不可计算问题。击碎了我们本节中给出的那些奇思妙想。看到这里,你是否想起了上一章附录中康托尔定理的证明?我们用极为类似的方法证明了任何集合,包括无穷集合的势都小于它的幂集的势。实际上,图灵停机问题让我们联想起了一大类有趣的逻辑悖论。

\section{罗素悖论}

% Eubulides of Miletus
悖论从古希腊时期就被人们发现了。上一章我们介绍了关于无穷和连续的芝诺悖论,而逻辑悖论是一类从严密逻辑导出矛盾结果的有趣问题。公元前四世纪,古希腊哲学家米利都的欧歩里德提出这样一个命题:“我现在说的是一句假话”,怎样判断这句话的真伪呢?

如果这句话是假话,那么它陈述的事实(正在说谎)就成了真的,因此矛盾。但如果这句话是真话,那么这句话的原话说正在说谎,因此它是假话,也产生了矛盾。不论欧歩里德说的是真是假,我们都将陷入矛盾中,这一著名的令人困惑的问题被人们称为“说谎者悖论”。

说谎者悖论还有一个两段体的变形,以对话的形式出现。例如:

\textbf{阿基里斯}:乌龟是个狡猾的家伙,总爱说谎,你听,它下面的话就是假的。

\textbf{乌龟}:亲爱的阿基里斯,诚实的你总是说真话。

乌龟的话到底是真是假呢?如果乌龟说了真话,也就是阿基里斯的陈述是真的。但阿基里斯说乌龟在撒谎,这就导致了矛盾。反之如果乌龟说的是假话,那么阿基里斯说的就是假的,于是乌龟说的这就话就应该为真。我们陷入了怪圈,无论乌龟的话是真是假,都会导致矛盾。

这种两段体式的说谎者悖论有时还以恶作剧的形式出现。你收到一张纸条,上面写着“背面是假的”,等你翻到纸条背面,却看到上面赫然写着“背面是真的”。到底哪面是真的呢?仔细分析下来,就会发现陷入了逻辑怪圈。

儿童故事中,也有不少这种悖论。有一则说狮子捉到了兔子,得意地说,如果你能猜中接下来我要干什么,我就放了你,要是猜错了,我就吃掉你。聪明的兔子说:“我猜你要吃掉我。”

如果狮子吃掉兔子,那说明兔子猜中了。这样狮子应该兑现承诺,放掉兔子。可是如果放掉兔子,这说明兔子猜错了。按道理狮子又应该吃掉兔子。狮子陷入了两难处境。既不能吃掉兔子,也不能不吃兔子。估计它只能发疯而让聪明的兔子溜走了。

传说古希腊的军队战胜了波斯,国王决心“优待俘虏”,让他们选择死亡的方式。俘虏可以说一句话,如果是真话,就被砍头,如果是假话,就被绞死。一个聪明的俘虏说:“我猜你要绞死我。”如果国王绞死了俘虏,说明他说了真话,可是这样,按照规则应该被砍头。但如果砍掉他的头,就和这个人讲的内容不符了,所以他说了假话。这样就应该被绞死。结果不论砍头还是绞死,国王的命令都没有被正确的执行。国王万般无奈,不仅释放了这个俘虏,还释放了所有其他人。

\begin{wrapfigure}{L}{0.5\textwidth}
 \centering
 \includegraphics[scale=0.17]{img/father-and-son.eps}
 \captionsetup{labelformat=empty}
 \caption{[德]埃$\cdot$奥$\cdot$卜劳恩《父与子》一则,1930年代}
 \label{fig:father-and-son}
\end{wrapfigure}

塞万提斯在他的伟大作品《堂$\cdot$吉诃德》中,也讲了一个有趣的悖论。有一位贵族的封地被一条大河分成了两半,河上有一座桥,桥的尽头有个绞架。这位贵族制定了一条法令:“过桥的人必须诚实声明他的目的,如果是真话就允许过桥,如果说谎,就判处绞刑,绞死在桥那边的绞架上。”结果有个人来这里发誓道,我过桥别无目的,就是想死在那个绞架上。怎样处置这个人呢?如果他说了真话,那么就应该放他过桥。可是这样这个人说的内容就不成立了,按照法令就应该绞死他。可是这样以来他说的话就成了真话,又应该放他过桥。

% https://en.wikipedia.org/wiki/Barber_paradox
和说谎者悖论同样著名的是理发师悖论。这是1919年由著名数学家、逻辑学家罗素罗素提出的。故事说村子里的理发师宣布:“他只给那些不给自己刮胡子的人刮胡子。”那么这位理发师是否给他自己刮胡子?如果他给自己刮胡子,那么按照他的规定,他就不应该给自己刮胡子。而如果他不给自己刮胡子,那么他就应该向自己提供服务,也就是给自己刮胡子。理发师这样就会陷入了困境。

罗素归纳总结了一系列悖论,并最终将它们形式化为当时集合论本质上的问题。人们现在一般将这类悖论称为罗素悖论。罗素最早在1901年发现了集合论的悖论。在康托尔的朴素集合论中,罗素考虑了任何集合是否属于它自身的问题。有些集合属于它本身,有些集合则不属于。例如所有茶匙的集合显然不是另一个茶匙,但所有不是茶匙的东西构成的集合显然也不是一个茶匙。罗素考虑了后者这类情况全体构成的集合。他构造了集合$R$,由所有不是自身元素的集合所组成。用形式化的定义表示就是:

\[
R = \{ x | x \notin x \}
\]

罗素接着思考,$R$是否属于$R$呢?根据逻辑中的排中律,一个元素或者属于一个集合,或者不属于一个集合。因此对于一个给定的集合,问它是否属于自己是有意义的。但是这个定义良好的,看似合理的问题却陷入了两难境地。

如果$R$属于$R$,那么根据$R$的定义,它只包含不属于自身的元素构成的集合,应该有$R$不属于$R$。反之,如果$R$不属于$R$,同样根据定义,它包含不属于自身的集合,又应该有$R$属于$R$。不管属于或不属于,都会导致矛盾。形式化的表达就是:

\[
R \in R \iff R \notin R
\]

这样罗素就明确表明了康托尔的集合论中存在悖论。

\begin{wrapfigure}{R}{0.3\textwidth}
 \centering
 \includegraphics[scale=0.5]{img/Russell.eps}
 \captionsetup{labelformat=empty}
 \caption{伯特兰$\cdot$罗素 1872-1970}
 \label{fig:Russell}
\end{wrapfigure}

%Monmouthshire
罗素1872年生于英国蒙茅茨郡的一个贵族家庭。两岁时母亲去世,三岁时父亲也去世。6岁时祖父也去世了,于是罗素和祖母生活在一起。祖母对他的童年和青少年时期的发展有过决定性的影响。她曾告诫罗素:“你不应该追随众人去做坏事”,罗素一生都努力遵循这条准则。

罗素少年时代未被送到学校去学习,而是在家接受教育。1883年开始,11岁的罗素跟随堂哥弗兰克学欧几里德几何。不久,罗素开始接触哲学思辨,并在宗教问题上,悄悄写下自己的想法在一家杂志发表。1890年罗素考入剑桥大学三一学院,大学前三年,他专攻数学,获数学荣誉学位考试的第七名。1894年,参加伦理学荣誉学位考试。完成研究论文《论几何学的基础》。在剑桥期间,他结识了当时的数学讲师怀特海等人。1895年,罗素在三一学院获得了研究员的职位。二十世纪初,他发现了著名的罗素悖论,并引发了一场关于数学基础的大讨论。其后十多年间,罗素投身于数学基础和数理逻辑的研究中。1920年,罗素应邀到中国讲学一年。足迹遍及中华南北,作了多场演讲。话题从数理逻辑到切中时弊的社会改造建议,在当时成为中国文化界的一件盛事。给我国哲学界以很大的影响。他的《西方哲学史》在我国的哲学爱好者中有着广泛的影响。

二十世纪50年代后,罗素从哲学转向国际政治。他反对核战争、主张核裁军。由于伸张民主和参加核裁军运动,罗素一生曾两次被捕入狱。其中第二次入狱时已经是89岁高龄。1950年罗素获得诺贝尔文学奖。委员会在授奖时称他为“当代理性和人道的最杰出代言人之一,西方自由言论和自由思想的无谓斗士。”

1970年2月2日,罗素在彭林德拉耶斯逝世,他的骨灰被撒在威尔士的群山之中。
% Russell died of influenza on 2 February 1970 at his home in Penrhyndeudraeth. His body was cremated in Colwyn Bay on 5 February 1970. In accordance with his will, there was no religious ceremony; his ashes were scattered over the Welsh mountains later that year.

\subsection{罗素悖论的影响}

罗素发现集合论基础的悖论后极为沮丧。他后来回忆道:“每天早晨,我面对一张白纸坐在那儿,除了短暂的午餐,我一整天都盯着那张白纸。常常在夜幕降临之际,仍是一片空白……似乎我整个余生很可能就消耗在这张白纸上。让人更烦恼的是,矛盾是平凡的。我的时间都花在这些似乎不值得考虑的事情上。”(\cite{HanXueTao16},第231页)罗素把他的发现告诉了数学家、逻辑学家弗雷格。当时弗雷格正在进行算术基础的建立工作,他的著作《算术的基本规律》已在付印中。弗雷格看到罗素悖论后非常沮丧,他写道:“一个科学家所遇到的最不合心意的事莫过于在他工作即将结束时,其基础崩溃了。罗素先生的一封信正好把我置于这个境地。”戴徳金也推迟了《什么是数的本质》一书的再版。罗素悖论涉及的是集合论中最基础的部分。由于集合论逐渐被大家接受,并进入了大多数数学分支,这使得人们对于数学和逻辑学的基本原理和有效性产生了怀疑。

\begin{Exercise}
\Question{我们可以用语言定义数,例如“最大的两位数”定义了99。定义一个集合,是所有不能用20个以内的字描述的数字。考虑这样一个元素:“不能用20个以内的字描述的最小数”,它是否属于这个集合?}
\Question{“这个世界上唯一不变的是变化”——这句话是否是罗素悖论?}
\Question{本章开头苏格拉底的话是否是罗素悖论?}
\end{Exercise}

\section{数学基础的分歧}

为了解决罗素悖论这一影响理性思维基础的问题。数学家们从1900年到1930年间持续进行讨论并各自提出了解决方案。数学在历史上长期被当作理性思维的真理,其绝对性和唯一性从未被引起怀疑和争论。在这一大讨论中,人们终于意识到,在不同的哲学观念下,可以存在不同的数学。

\subsection{逻辑主义}

\begin{wrapfigure}{L}{0.3\textwidth}
 \centering
 \includegraphics[scale=0.9]{img/Frege.eps}
 \captionsetup{labelformat=empty}
 \caption{戈特洛布$\cdot$弗雷格(1848-1925)}
 \label{fig:Frege}
\end{wrapfigure}

逻辑主义的早期代表人物是弗雷格。他认为数学的基础并不是数,算术理论可以建立在逻辑的基础上。弗雷格把朴素集合论看作是逻辑的一部分。他做的第一件工作是利用逻辑来定义自然数。我们知道数具有抽象的含义。3可以代表3个人、三个鸡蛋、图形中的三角等等,这些类\footnote{弗雷格的工作在康托尔之前,他当时使用了“类”(class)一词。康托尔后来使用了德语中的“集合”。}都有三个元素,用哪一个来代表自然数3呢?弗雷格的意见是:全部。即所有能和上述类一一对应的类所组成的,无穷的,抽象的类来定义自然数3。弗雷格的这一定义看起来有些复杂,但是很了不起。它突破了文化背景的限制。不管你是使用何种语言,何种符号,按照弗雷格的方法,对数字3的理解都不会有歧义。因为佛雷格的定义中根本不需要任何符号。这样佛雷格就定义了数——它是所有类的类。接下来佛雷格借助这一定义和逻辑理论建立了自然数的理论,进而形成了逻辑化的算术理论。再进一步,弗雷格打算从利用逻辑发展出除几何以外的全部数学。这就是他在《算术的基本规律》一书中打算完成的事情。由于弗雷格坚信逻辑的原则是完全可靠的,这样他的工作一旦完成,数学“就被固定在一个永恒的基础上了。”

我们知道接下来发生了什么。在《算术的基本规律》正在付印时,罗素的信“及时”寄到了。罗素悖论让弗雷格陷入了困惑。他工作的基础——利用逻辑定义数的概念——恰好是关于所有类的类。这样的定义直接导致了悖论的出现。弗雷格动摇了,并最终放弃了他的逻辑主义立场。

罗素接过了逻辑主义的火炬,积极投身于数学基础的重建和悖论的解决工作中。罗素坚信数学就是逻辑,逻辑就是数学。他形成这样的思想,意大利数学家皮亚诺起到了重要的作用。罗素后来回忆道:“在我学术生命中最重要的一年是1900年,而这一年中最重要的事是我去巴黎参加国际哲学会议……皮亚诺和他的学生们在一切讨论中所表现出来的,为他人所没有的精确性,给了我深刻的印象。”他和怀特海两人每天讨论数学的基本概念,经过艰苦的工作终于写出了著名的《数学原理》\footnote{罗素和怀特海希望通过书的名字向牛顿致敬。因为牛顿的著作名为《自然哲学中的数学原理》}三卷本分别在1910年到1913年间出版。可以说是数理逻辑的经典之作。为了解决悖论,罗素指出一切悖论都源于某种“恶性循环”,而恶性循环源于某种不合法的集体,具体的说就是集体的整体这一概念。所以要想消除悖论、避免恶性循环,凡是涉及一个集体的整体的对象,它本身不能是该集体的成员。从这一思想出发,罗素提出了“分支类型论”。

\begin{wrapfigure}{R}{0.4\textwidth}
 \centering
 \includegraphics[scale=0.4]{img/Whitehead.eps}
 \captionsetup{labelformat=empty}
 \caption{阿尔弗雷德$\cdot$怀特海(1861-1947)}
 \label{fig:Whitehead}
\end{wrapfigure}

分支类型论将集合进行了层次划分,定义域中的对象个体属于第0类;个体的集合属于第1类;第1类中的个体的集合,也就是集合的集合,属于第2类……每一集合都必须从属确定的类。而命题中的对象必须从属于它所在的等级。这样做可以有效地消除悖论,但使用起来极其繁琐不便。《数学原理》第一卷知道第363页才推出数字1的定义。庞加莱挖苦道:“这是一个可亲可佩的定义,它献给那些从来不知道1的人。”使用分支类型论,所有的工作只能在各自的等级上进行,整数在整数的等级上,有理数在有理数的等级上,我们不能把$n/1$和$n$混为一谈。更为严重的是:“所有实数……”这样的命题不合法类,因为它涉及了集合中不同的层次。

但最有争议的地方是“无穷公理”、“选择公理”、“约化公理”的使用。为了处理自然数以及更为复杂的实数和超限数,罗素和怀特海引入公理来承认无穷的存在。他们也承认可以从非空集合,甚至无穷集合中选择元素组成新的集合。这两条有争议的公理在集合论中也存在,但最难以让其他数学家接受的是约化公理。为了支持数学归纳法,约化公理认为任何较高层次的一个命题与一个层次为0的命题等价。约化公理激起了反对,因为它显得太任意了。1909年庞加莱说:约化公理比数学归纳法更靠不住,更含糊不清。

后来连罗素自己也动摇了:“从严格的逻辑化来看,我找不出任何理由来相信约化公理是逻辑必然的,这就是说,它在所有可能的世界中都是真的。因此,在逻辑体系中,承认这个公理是个缺憾,即使从经验来看是真的。”\cite{M-Kline-2007}

\subsection{直觉主义}

\begin{wrapfigure}{L}{0.4\textwidth}
 \centering
 \includegraphics[scale=0.28]{img/Brouwer.eps}
 \captionsetup{labelformat=empty}
 \caption{布劳威尔(1881-1966)}
 \label{fig:Whitehead}
\end{wrapfigure}

%图片来自The Low Countries, Arts and Society in Flanders and the netherlands. A year book 1998-99

与逻辑主义同时,另一群称为直觉主义的数学家使用了截然不同、完全相反的方法来重建数学的基础。逻辑主义的先驱是克罗内克,其代表人物是荷兰数学家布劳威尔。布劳威尔于1881年生于荷兰鹿特丹附近的小镇奥弗希。很早就显露出与众不同的才华。1897年他考入阿姆斯特丹大学攻读数学。在读大学时,他获得了关于四维空间连续运动的某些结果,并发表在阿姆斯特丹皇家科学院报告集上,在当大学生时,通过自己的刻苦钻研,更由于受到曼诺利(Mannoury)教授一系列启迪性讲座的启发,布劳威尔接触到了拓扑学和数学基础,并且终生钟爱它们。

受到希尔伯特在巴黎的第二届国际数学家大会的讲演的影响,布劳威尔从1907年到1913年进行了大量拓扑学的研究。建立布劳威尔不动点定理是他的突出贡献。这个定理表明:在二维球面上,任意映到自身的一一连续映射,必定至少有一个点是不变的。他把这一定理推广到高维球面。尤其是在n维球内映到自身的任意连续映射至少有一个不动点。1910年,布劳威尔证明了维数的拓扑不变性。1913年,他给出了拓扑空间维数的严格定义。由于布劳威尔在拓扑学上的出色成就,他被推选为荷兰皇家科学院院士。

在攻读博士学位时,布劳威尔以极大的热情关注着罗素庞加莱关于数学的逻辑基础的论战,并以此为题写成他的博士论文。总的说来,他倾向于庞加莱的观点,反对罗素和希尔伯特关于数学基础的思想。但是,他又极不同意庞加莱关于数学存在性的说法。他认为,庞加莱的办法不能排除悖论。为此,他在博士论文“论数学基础”中开始建立直觉主义的数学哲学。1966年,85岁的布劳威尔不幸死于车祸。

\subsection{形式主义}

\section{公理集合论}

\section{哥德尔不完全性定理}

\section{不完全性定理的证明}

\section{万能的程序与对角线证明}

\section{尾声}
理性思维的边界、敬畏之心。

\begin{wrapfigure}{R}{0.5\textwidth}
%\begin{figure}[htbp]
 \centering
 \includegraphics[scale=1]{img/Escher-Dragon.eps}
 %\captionsetup{labelformat=empty}
 \caption{埃舍尔《龙》}
 \label{fig:Escher-Dragon}
%\end{figure}
\end{wrapfigure}


\ifx\wholebook\relax \else
\begin{thebibliography}{99}

\bibitem{Gatys-2015}
Leon A. Gatys, Alexander S. Ecker, Matthias Bethge. ``A Neural Algorithm of Artistic Style.'' 2015. arXiv:1508.06576 [cs.CV] IEEE Conference on Computer Vision and Pattern Recognition (CVPR) 2017.

\bibitem{GuSen-2012}
顾森 《思考的乐趣——Matrix67数学笔记》 人民邮电出版社,2012年,ISBN: 9787115275868

\bibitem{SICP}
Harold Abelson, Gerald Jay Sussman, Julie Sussman 著 裘宗燕 译 ``计算机程序的构造和解释(原书第二版)''. 北京 机械工业出版社 2004年 ISBN: 7-111-13510-5

\bibitem{HanXueTao16}
韩雪涛 ``数学悖论与三次数学危机''. 人民邮电出版社. 2016, ISBN: 9787115430434

\bibitem{M-Kline-2007}
[美] M$\cdot$克莱因 著 李宏魁 译 ``数学:确定性的丧失'' 湖南科学技术出版社,2007年4月 ISBN: 978-7-5357-1857-0
% Morris Kline ``Mathematics: The Loss of Certainty''. Oxford University Press, 1980.

\end{thebibliography}

\expandafter\enddocument
%\end{document}

\fi
