\ifx\wholebook\relax \else

\documentclass{article}

\input{../common-zh-cn.tex}

\setcounter{page}{1}

\begin{document}

\title{悖论}

\author{刘新宇
\thanks{{\bfseries 刘新宇} \newline
  Email: liuxinyu95@gmail.com \newline}
  }

\maketitle
\fi

\markboth{悖论}{编程的数学原理}

\ifx\wholebook\relax
\chapter{悖论}
\numberwithin{Exercise}{chapter}
\fi

\epigraph{除了自己的无知,我什么都不懂。}{——苏格拉底}

\begin{wrapfigure}{R}{0.5\textwidth}
 \centering
 \includegraphics[scale=0.3]{img/Escher-Waterfall-1961.eps}
 \captionsetup{labelformat=empty}
 \caption{埃舍尔《瀑布》1961}
 \label{fig:Escher-Waterfall}
\end{wrapfigure}

1996年,第26界国际奥林匹克运动会正在美国的亚特兰大举行。来自世界各地的选手在速度、力量、技巧上展开竞赛挑战人类的极限。与此同时,还进行着另一场有趣的竞赛。超级计算机“深蓝”与人类的国际象棋世界冠军卡斯帕罗夫展开了对抗赛。比赛结果是深蓝2胜4负输给了人类象棋冠军。翌年,改进的深蓝再次向卡斯帕罗夫发起挑战。5月11日,计算机在正常时限的比赛中首次击败了卡斯帕罗夫。总比分1胜2负3平。深蓝计算机重1270公斤,有32个微处理器,每秒钟可以计算2亿步。为了能让"深蓝”挑战人类冠军,设计小组输入了一百多年来优秀棋手的对局两百多万对局。人类用智慧创造的机器,在人类骄傲的智慧领域首次击败了人类——这一结局引发了关注、恐惧、和激烈的讨论。

在当时,人们普遍认为,这是人工智能的一大进步。尽管在国际象棋上取得了巨大进步,但是在围棋上,计算机和人类仍存在巨大差距。对于国际象棋来说,棋盘8行8列,32枚棋子。计算机要在$10^{123}$这样巨大的博弈树中进行搜索。即使深蓝美妙能算2亿步,遍历博弈树仍需要近$10^{107}$年。为此深蓝的设计小组通过计算机程序缩小了搜索空间,使得深蓝能够搜索当前棋局后面的12步棋。而一般好的人类棋手,大约只能估计到10步左右。但是围棋的棋盘有19行19列,共361个格点上可以放黑色或者白色的棋子。博弈树的规模为$10^{360}$,远远超越国际象棋。所以在之后的一段时间里,人们仍然不相信计算机可以挑战我们。

%\begin{wrapfigure}{R}{0.5\textwidth}
\begin{figure}[htbp]
 \centering
 \includegraphics[scale=0.4]{img/Deep-blue-1997.eps}
 \captionsetup{labelformat=empty}
 \caption{卡斯帕罗夫在与深蓝对弈,图片原载《科学美国人杂志》}
 \label{fig:Deep-blue-1997}
\end{figure}
%\end{wrapfigure}

时光匆匆过去了20年,2016年,计算机程序Alpha-Go向人类的围棋大师展开了挑战。韩国的九段棋手李世石以1比4输掉了比赛。一年后,再次以3局全胜战胜了中国棋手柯洁。被人们认为是人工智能游戏“圣杯”的围棋终于被攻破了。作为人类,我们的心情很复杂。即使是从事智力工作的程序员群体也感到了来自机器的压力——我们是否会被机器取代?

%University of Tubingen, Germany
% Leon A. Gatys, Alexander S. Ecker Matthias Bethge
传统上我们认为,艺术文学等领域,涉及人们的文化背景、感情和性格因素,是无法被机器替代的。2015年,德国斯图加特以南40公里的小镇图宾根大学大学的盖提斯,埃克,贝特格利用机器学习人类艺术家的风格,把图宾根镇的风景照片变换成了不同风格的艺术画作。

%\begin{wrapfigure}{R}{0.5\textwidth}
\begin{figure}[htbp]
 \centering
 \includegraphics[scale=0.85]{img/style-transfer.eps}
 \captionsetup{labelformat=empty}
 \caption{机器学习产生的不同艺术风格的画作:A,图宾根镇的风景照片;B,英国画家透纳1810年的原作《运输船遇难》和透纳风格的画作;C,荷兰后印象派画家梵高1889年的原作《星空》和梵高风格的画作;D,挪威表现主义画家爱德华$\cdot$蒙克1893年的原作《呐喊》和蒙克风格的画作;E,西班牙现代艺术家毕加索1910年的原作《坐着的裸女》和毕加索风格的画作;F,俄罗斯抽象艺术先驱画家康定斯基1913年的原作《构成第七号》和康定斯基风格的画作}
 \label{fig:Deep-blue-1997}
\end{figure}
%\end{wrapfigure}


回顾人类面临人工智能挑战的情况,国际象棋、围棋、绘画、医疗影像判断等。
提出计算机能否战胜人类的问题

\section{计算的边界}

图灵停机问题

停机问题类比,并引出罗素悖论

\section{罗素悖论}

\section{数学基础的分歧}

逻辑主义

直觉主义

形式主义

\section{公理集合论}

\section{哥德尔不完全性定理}

\section{不完全性定理的证明}

\section{万能的程序与对角线证明}

\section{尾声}
理性思维的边界。

埃舍尔的龙

\ifx\wholebook\relax \else
\begin{thebibliography}{99}

\bibitem{De-linfini-2018}
[法] 让-皮埃尔$\cdot$卢米涅,马克$\cdot$拉雪茨-雷 著,孙展 译. 从无穷开始——科学的困惑与疆界. 人民邮电出版社. 2018. ISBN: 9787115479198

\bibitem{Noguchi2007}
[日] 野口哲也 著,刘慧 韩丽红 译. 数学原来可以这样学. 湖南人民出版社. 2014. ISBN: 9787556100897
% Tetsunori Noguchi. SUGAKUTEKI SENSE GA MINITUKU RENSHUCHO.

\bibitem{Wikipedia-Googol}
Wikipedia. ``Googol''. \url{https://en.wikipedia.org/wiki/Googol}

\bibitem{Wikipedia-Zeno}
Wikipedia. ``Zeno's Paradoxes''. \url{https://en.wikipedia.org/wiki/Zeno's_paradoxes}

\bibitem{HanXueTao16}
韩雪涛 ``数学悖论与三次数学危机''. 人民邮电出版社. 2016, ISBN: 9787115430434

\bibitem{Elements}
[古希腊] 欧几里得 著,兰纪正 朱恩宽 译,梁宗巨 张毓新 徐伯谦 校订 ``几何原本''. 译林出版社. 2014, ISBN: 9787544750066

\bibitem{M-Kline-2007}
[美] M$\cdot$克莱因 著 李宏魁 译 ``数学:确定性的丧失'' 湖南科学技术出版社,2007年4月 ISBN: 978-7-5357-1857-0
% Morris Kline ``Mathematics: The Loss of Certainty''. Oxford University Press, 1980.

\bibitem{Stepanov}
Stepanov and Rose. ``数学与泛型编程''. 爱飞翔译,机械工业出版社。ISBN: 978-7-111-57658-7. 2017.

\bibitem{GEB}
[美]候世达 ``哥德尔、埃舍尔、巴赫——集异壁之大成''. 商务印书馆 1996. ISBN: 978-7-100-01323-9

\bibitem{GCH}
张锦文,王雪生 ``连续统假设''. 世界数学名题欣赏丛书。辽宁教育出版社 1988. ISBN: 7-5382-0436-9/G$\cdot$445

\end{thebibliography}

\expandafter\enddocument
%\end{document}

\fi
