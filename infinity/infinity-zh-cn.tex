\ifx\wholebook\relax \else

\documentclass{article}

\input{../common-zh-cn.tex}

\setcounter{page}{1}

\begin{document}

\title{无穷}

\author{刘新宇
\thanks{{\bfseries 刘新宇} \newline
  Email: liuxinyu95@gmail.com \newline}
  }

\maketitle
\fi

\markboth{无穷}{编程的数学原理}

\ifx\wholebook\relax
\chapter{无穷}
\numberwithin{Exercise}{chapter}
\fi

\epigraph{我明白了,但我不相信。}{——理查德$\cdot$戴得金}

% "I see it, but I don't believe it." -- Richard Dedekind

\begin{wrapfigure}{R}{0.5\textwidth}
 \centering
 \includegraphics[scale=0.3]{img/circle-limit-IV-1960.eps}
 \captionsetup{labelformat=empty}
 \caption{埃舍尔《圆极限$\cdot$4》(又名天使与恶魔)1960}
 \label{fig:Penrose-triangle}
\end{wrapfigure}

不知在多久以前,我们的祖先仰望星空,面对浩瀚的星河,由衷地发出感叹,我们所在的世界究竟有多大?
作为智慧的生命,我们的思维超越自我,超越地球,超越宇宙,不断思考着无穷的概念。我们的祖先是从具体的事物中抽象出数了的概念。
例如狩猎得到的三头羊,采集得到三个果实,烧制了三个陶罐,进而得到抽象的数字三来代表任何三个东西。起初的数字大小有限,能够满足日常生活、狩猎、劳作。
随着文明的发展,我们开始进行贸易活动,出于记账的要求,所需要的数字逐渐变大。人们发展出种种计数系统,来掌握更大的数字。
终于,我们提出问题:最大的数是什么?对于这个问题,人们采用两种不同的态度。一种认为,这个问题没有意义,在古代,
掌握千百万这样的数已经足够生活中使用了。我们无需了解生活中用不上的大数。例如我们可以认为世界上沙子的数目是无穷的。
在古希腊,一万曾被认为是一个十分巨大的数,人们称它为murias,最终变成了myriad一词,意为“无数”\cite{De-linfini-2018}。无独有偶,佛教中也用“恒河沙数”来表达大到无法计算的数。在大乘佛教经典《金刚经》中,佛陀说:“以七宝满尔所恒河沙数三千大世界,以用布施。”另一种则不这么想,阿基米德,这位古希腊伟大的数学家认为,即使是充满全宇宙的沙子数目,也可以用一个数代表。在阿基米德在他的著作《数沙者》开篇中说:

“格朗王,有人认为沙子的数量是无穷的。我所说的沙子,并不单单地指叙拉古附近和西西里岛其余地方的沙子,还包括地球上所有角落能找到的沙子,无论那里有人还是无人居住。另一些人虽然承认沙子的数量并不是无穷大的,但他们认为,我们不可能写出一个足够大的数,使它在数量上超过地球上全部沙子所代表的数量。如果想象一个和地球体积同样大的沙体,而且要从地球上的大海和谷底算起,知道最高山峰的高度都填满沙子,这些人恐怕就更加肯定,世界上不可能有如此之大的数,可以用来表示堆积起这一巨大沙体所需要的沙子的数量。但是,我将向您证明,通过一系列几何张明——您之后也可以照着做——我命名了一些数字,写在我给宙克西珀的手稿中。其中一些数字不仅超过了以我刚刚描述过的方式填充地球所需要的沙子的数量,甚至超过了填充整个宇宙所需要的沙子数量。”

\begin{figure}[htbp]
%\begin{wrapfigure}{R}{0.4\textwidth}
 \centering
 \includegraphics[scale=0.5]{img/Archimedes.eps}
 \captionsetup{labelformat=empty}
 \caption{《数沙者》,封面的阿基米德像是意大利画家多米尼克$\cdot$费蒂1620年创作的。}
 \label{fig:Archimedes}
%\end{wrapfigure}
\end{figure}

阿基米德认为填满宇宙“只”需要$10^{63}$粒沙子。这个宇宙的含义是指恒星天球,大约为两万倍地球的半径。今天我们知道可观测宇宙的尺寸大约为460亿光年,大约包含$3 \times 10^{74}$个原子\footnote{一说为$10^{80}$到$10^{87}$个基本粒子。}。在古希腊的时代,阿基米德的想法无疑是天才的,这几乎是无穷的具体化。我们从语言中,可以看到许多表达大数的单位的词语。例如下表是汉语中的大单位,从“兆”以后,每增加一万倍,就有一个对应的单位。(\cite{Noguchi2007},第31页)

\begin{center}
% for Pinyin tones: \={a}, \'{a}, \v{}, \.{a}
\begin{tabular}{|l|r|l|r|}
\hline
京            & $10^{16}$ & 载            & $10^{44}$ \\
\hline
垓(g\={a}i)   & $10^{20}$ & 极            & $10^{48}$ \\
\hline
秭(z\v{i})    & $10^{24}$ & \textbf{恒河沙}  & $10^{52}$ \\
\hline
穰(r\'{a}ng)  & $10^{28}$ & 阿僧祗(zh\={i})  & $10^{56}$ \\
\hline
沟            & $10^{32}$ & 那由他        & $10^{60}$ \\
\hline
涧            & $10^{36}$ & 不可思议      & $10^{64}$ \\
\hline
正            & $10^{40}$ & 无量大数      & $10^{68}$ \\
\hline
\end{tabular}
\end{center}

可以看到,汉语中这些大单位词汇,由许多来自佛教。包括恒河沙,它表示1后面跟着52个0。英语中的大单位如下表。从一开始,每增加一千倍就有一个对应的单位。这种万进位和千进位的不同,也是文化上的一种差异。

\begin{center}
\begin{tabular}{|l|r|l|r|l|r|}
\hline
thousand & $10^{3}$ & quattuordecillion & $10^{45}$ & octovigintillion & $10^{87}$ \\
\hline
million & $10^{6}$ & quindecillion & $10^{48}$ & novemvigintillion & $10^{90}$ \\
\hline
billion & $10^{9}$ & sexdecillion & $10^{51}$ & trigintillion & $10^{93}$ \\
\hline
trillion  & $10^{12}$ & septdecillion & $10^{54}$ & untrigintillion & $10^{96}$ \\
\hline
quadrillion  & $10^{15}$ & octodecillion & $10^{57}$ & duotrigintillion & $10^{99}$ \\
\hline
quintillion  & $10^{18}$ & novemdecillion & $10^{60}$ & \textbf{googol} & $10^{100}$ \\
\hline
sexillion    & $10^{21}$ & vigintillion & $10^{63}$ & & \\
\hline
septillion   & $10^{24}$ & unvigintillion & $10^{66}$ & & \\
\hline
octillion    & $10^{27}$ & duovigintillion & $10^{69}$ & & \\
\hline
noniliion  & $10^{30}$ & trevigintillion & $10^{72}$ & & \\
\hline
decillion  & $10^{33}$ & quattuorvigintillion & $10^{75}$ & & \\
\hline
undecillion   & $10^{36}$ & quinvigintillion & $10^{78}$ & & \\
\hline
duodecillion  & $10^{39}$ & sexvigintillion & $10^{81}$ & & \\
\hline
tredecillion  & $10^{42}$ & seprvigintillion & $10^{84}$ & & \\
\hline
\end{tabular}
\end{center}

表中最后一个大单位古格尔(googol)是在1920年由9岁的米尔顿$\cdot$西洛塔(Milton Sirotta)想出的名字。这个数字是1后面跟着100个零。著名的互联网公司谷歌的名字就来自它\cite{Wikipedia-Googol}。

超越一切具体大数的无穷是否存在不仅是一个数学问题,还是一个哲学问题。无穷大还意味着它的另一面——无穷小。古代中国的哲学家庄子在《天下篇》中说:“一尺之棰,日取其半,万世不竭。”,古希腊的哲学家,埃利亚学派的芝诺(Zeno of Elea)提出了著名的四个悖论,它们都与无穷有关。

\begin{wrapfigure}{L}{0.4\textwidth}
 \centering
 \includegraphics[scale=1.0]{img/Zeno.eps}
 \captionsetup{labelformat=empty}
 \caption{芝诺,约490BC - 425BC}
 \label{fig:Zeno-of-Elea}
\end{wrapfigure}

欧几里得素数有无穷多的证明。

\begin{wrapfigure}{L}{0.4\textwidth}
 \centering
 \includegraphics[scale=1.0]{img/circle-limit-III-1959.eps}
 \captionsetup{labelformat=empty}
 \caption{埃舍尔《圆极限$\cdot$3》1959}
 \label{fig:Penrose-triangle}
\end{wrapfigure}

% Mathematicians aren't satisfied because they know there are no solutions up to four million or four billion, they really want to know that there are no solutions up to infinity. -- Andrew Wiles

\ifx\wholebook\relax \else
\begin{thebibliography}{99}

\bibitem{De-linfini-2018}
[法] 让-皮埃尔$\cdot$卢米涅,马克$\cdot$拉雪茨-雷 著,孙展 译. 从无穷开始——科学的困惑与疆界. 人民邮电出版社. 2018. ISBN: 9787115479198

\bibitem{Noguchi2007}
[日] 野口哲也 著,刘慧 韩丽红 译. 数学原来可以这样学. 湖南人民出版社. 2014. ISBN: 9787556100897
% Tetsunori Noguchi. SUGAKUTEKI SENSE GA MINITUKU RENSHUCHO.

\bibitem{Wikipedia-Googol}
Wikipedia. ``Googol''. \url{https://en.wikipedia.org/wiki/Googol}

\end{thebibliography}

\expandafter\enddocument
%\end{document}

\fi
