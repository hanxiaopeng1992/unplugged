\ifx\wholebook\relax \else

\documentclass{article}

\input{../common-zh-cn.tex}

\setcounter{page}{1}

\begin{document}

\title{无穷}

\author{刘新宇
\thanks{{\bfseries 刘新宇} \newline
  Email: liuxinyu95@gmail.com \newline}
  }

\maketitle
\fi

\markboth{无穷}{编程的数学原理}

\ifx\wholebook\relax
\chapter{无穷}
\numberwithin{Exercise}{chapter}
\fi

% \epigraph{3表示2+1,4表示3+1。所以接下来(虽然证明很长),4等于2+2。因此, 数学知识不再是神秘的。}{——罗素}

% ... mathematical knowledge ... is, in fact, merely verbal knowledge. "3" means "2+1", and "4" means "3+1". Hence it follows (though the proof is long) that "4" means the same as "2+2". Thus mathematical knowledge ceases to be mysterious.  -- Bertrand Russell

\begin{wrapfigure}{R}{0.4\textwidth}
 \centering
 \includegraphics[scale=0.35]{img/circle-limit-IV-1960.eps}
 \captionsetup{labelformat=empty}
 \caption{埃舍尔《圆极限$\cdot$4》(又名天使与恶魔)1960}
 \label{fig:Penrose-triangle}
\end{wrapfigure}

\begin{wrapfigure}{L}{0.4\textwidth}
 \centering
 \includegraphics[scale=1.0]{img/circle-limit-III-1959.eps}
 \captionsetup{labelformat=empty}
 \caption{埃舍尔《圆极限$\cdot$3》1959}
 \label{fig:Penrose-triangle}
\end{wrapfigure}

\ifx\wholebook\relax \else
\begin{thebibliography}{99}

\end{thebibliography}

\expandafter\enddocument
%\end{document}

\fi
