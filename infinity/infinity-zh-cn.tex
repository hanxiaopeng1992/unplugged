\ifx\wholebook\relax \else

\documentclass{article}

\input{../common-zh-cn.tex}

\setcounter{page}{1}

\begin{document}

\title{无穷}

\author{刘新宇
\thanks{{\bfseries 刘新宇} \newline
  Email: liuxinyu95@gmail.com \newline}
  }

\maketitle
\fi

\markboth{无穷}{编程的数学原理}

\ifx\wholebook\relax
\chapter{无穷}
\numberwithin{Exercise}{chapter}
\fi

\epigraph{我明白了,但我不相信。}{——理查德$\cdot$戴得金}

% "I see it, but I don't believe it." -- Richard Dedekind

\begin{wrapfigure}{R}{0.4\textwidth}
 \centering
 \includegraphics[scale=0.35]{img/circle-limit-IV-1960.eps}
 \captionsetup{labelformat=empty}
 \caption{埃舍尔《圆极限$\cdot$4》(又名天使与恶魔)1960}
 \label{fig:Penrose-triangle}
\end{wrapfigure}

不知在多久以前,我们的祖先仰望星空,面对浩瀚的星河,由衷地发出感叹,我们所在的世界究竟有多大?
作为智慧的生命,我们的思维超越自我,超越地球,超越宇宙,不断思考着无穷的概念。我们的祖先是从具体的事物中抽象出数了的概念。
例如狩猎得到的三头羊,采集得到三个果实,烧制了三个陶罐,进而得到抽象的数字三来代表任何三个东西。起初的数字大小有限,能够满足日常生活、狩猎、劳作。
随着文明的发展,我们开始进行贸易活动,出于记账的要求,所需要的数字逐渐变大。人们发展出种种计数系统,来掌握更大的数字。
终于,我们提出问题:最大的数是什么?对于这个问题,人们采用两种不同的态度。一种认为,这个问题没有意义,在古代,
掌握千百万这样的数已经足够生活中使用了。我们无需了解生活中用不上的大数。例如我们可以认为世界上沙子的数目是无穷的。
在古希腊,一万曾被认为是一个十分巨大的数,人们称它为murias,最终变成了myriad一词,意为“无数”\cite{Del-linfini-2018}。另一种则不这么想,阿基米德,这位古希腊伟大的数学家认为,即使是充满全宇宙的沙子数目,也可以用一个数代表。在阿基米德在他的著作《数沙者》开篇中说:

“格朗王,”

\begin{wrapfigure}{L}{0.4\textwidth}
 \centering
 \includegraphics[scale=1.0]{img/circle-limit-III-1959.eps}
 \captionsetup{labelformat=empty}
 \caption{埃舍尔《圆极限$\cdot$3》1959}
 \label{fig:Penrose-triangle}
\end{wrapfigure}

% Mathematicians aren't satisfied because they know there are no solutions up to four million or four billion, they really want to know that there are no solutions up to infinity. -- Andrew Wiles

\ifx\wholebook\relax \else
\begin{thebibliography}{99}

\bibitem{De-linfini-2018}
[法] 让-皮埃尔$\cdot$卢米涅,马克$\cdot$拉雪茨-雷 著,孙展 译. 从无穷开始——科学的困惑与疆界. 人民邮电出版社. 2018. ISBN: 9787115479198

\end{thebibliography}

\expandafter\enddocument
%\end{document}

\fi
