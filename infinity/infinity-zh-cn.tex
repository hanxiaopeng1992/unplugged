\ifx\wholebook\relax \else

\documentclass{article}

\input{../common-zh-cn.tex}

\setcounter{page}{1}

\begin{document}

\title{无穷}

\author{刘新宇
\thanks{{\bfseries 刘新宇} \newline
  Email: liuxinyu95@gmail.com \newline}
  }

\maketitle
\fi

\markboth{无穷}{编程的数学原理}

\ifx\wholebook\relax
\chapter{无穷}
\numberwithin{Exercise}{chapter}
\fi

\epigraph{我明白了,但我不相信。}{——理查德$\cdot$戴得金}

% "I see it, but I don't believe it." -- Richard Dedekind

\begin{wrapfigure}{R}{0.5\textwidth}
 \centering
 \includegraphics[scale=0.3]{img/circle-limit-IV-1960.eps}
 \captionsetup{labelformat=empty}
 \caption{埃舍尔《圆极限$\cdot$4》(又名天使与恶魔)1960}
 \label{fig:Penrose-triangle}
\end{wrapfigure}

不知在多久以前,我们的祖先仰望星空,面对浩瀚的星河,由衷地发出感叹,我们所在的世界究竟有多大?
作为智慧的生命,我们的思维超越自我,超越地球,超越宇宙,不断思考着无穷的概念。我们的祖先是从具体的事物中抽象出数了的概念。
例如狩猎得到的三头羊,采集得到三个果实,烧制了三个陶罐,进而得到抽象的数字三来代表任何三个东西。起初的数字大小有限,能够满足日常生活、狩猎、劳作。
随着文明的发展,我们开始进行贸易活动,出于记账的要求,所需要的数字逐渐变大。人们发展出种种计数系统,来掌握更大的数字。
终于,我们提出问题:最大的数是什么?对于这个问题,人们采用两种不同的态度。一种认为,这个问题没有意义,在古代,
掌握千百万这样的数已经足够生活中使用了。我们无需了解生活中用不上的大数。例如我们可以认为世界上沙子的数目是无穷的。
在古希腊,一万曾被认为是一个十分巨大的数,人们称它为murias,最终变成了myriad一词,意为“无数”\cite{De-linfini-2018}。无独有偶,佛教中也用“恒河沙数”来表达大到无法计算的数。在大乘佛教经典《金刚经》中,佛陀说:“以七宝满尔所恒河沙数三千大世界,以用布施。”另一种则不这么想,阿基米德,这位古希腊伟大的数学家认为,即使是充满全宇宙的沙子数目,也可以用一个数代表。在阿基米德在他的著作《数沙者》开篇中说:

“格朗王,有人认为沙子的数量是无穷的。我所说的沙子,并不单单地指叙拉古附近和西西里岛其余地方的沙子,还包括地球上所有角落能找到的沙子,无论那里有人还是无人居住。另一些人虽然承认沙子的数量并不是无穷大的,但他们认为,我们不可能写出一个足够大的数,使它在数量上超过地球上全部沙子所代表的数量。如果想象一个和地球体积同样大的沙体,而且要从地球上的大海和谷底算起,知道最高山峰的高度都填满沙子,这些人恐怕就更加肯定,世界上不可能有如此之大的数,可以用来表示堆积起这一巨大沙体所需要的沙子的数量。但是,我将向您证明,通过一系列几何张明——您之后也可以照着做——我命名了一些数字,写在我给宙克西珀的手稿中。其中一些数字不仅超过了以我刚刚描述过的方式填充地球所需要的沙子的数量,甚至超过了填充整个宇宙所需要的沙子数量。”

\begin{figure}[htbp]
%\begin{wrapfigure}{R}{0.4\textwidth}
 \centering
 \includegraphics[scale=0.5]{img/Archimedes.eps}
 \captionsetup{labelformat=empty}
 \caption{《数沙者》,封面的阿基米德像是意大利画家多米尼克$\cdot$费蒂1620年创作的。}
 \label{fig:Archimedes}
%\end{wrapfigure}
\end{figure}

阿基米德认为填满宇宙“只”需要$10^{63}$粒沙子。这个宇宙的含义是指恒星天球,大约为两万倍地球的半径。今天我们知道可观测宇宙的尺寸大约为460亿光年,大约包含$3 \times 10^{74}$个原子\footnote{一说为$10^{80}$到$10^{87}$个基本粒子。}。在古希腊的时代,阿基米德的想法无疑是天才的,这几乎是无穷的具体化。我们从语言中,可以看到许多表达大数的单位的词语。例如下表是汉语中的大单位,从“兆”以后,每增加一万倍,就有一个对应的单位。(\cite{Noguchi2007},第31页)

\begin{center}
% for Pinyin tones: \={a}, \'{a}, \v{}, \.{a}
\begin{tabular}{|l|r|l|r|}
\hline
京            & $10^{16}$ & 载            & $10^{44}$ \\
\hline
垓(g\={a}i)   & $10^{20}$ & 极            & $10^{48}$ \\
\hline
秭(z\v{i})    & $10^{24}$ & \textbf{恒河沙}  & $10^{52}$ \\
\hline
穰(r\'{a}ng)  & $10^{28}$ & 阿僧祗(zh\={i})  & $10^{56}$ \\
\hline
沟            & $10^{32}$ & 那由他        & $10^{60}$ \\
\hline
涧            & $10^{36}$ & 不可思议      & $10^{64}$ \\
\hline
正            & $10^{40}$ & 无量大数      & $10^{68}$ \\
\hline
\end{tabular}
\end{center}

可以看到,汉语中这些大单位词汇,由许多来自佛教。包括恒河沙,它表示1后面跟着52个0。英语中的大单位如下表。从一开始,每增加一千倍就有一个对应的单位。这种万进位和千进位的不同,也是文化上的一种差异。

\begin{center}
\begin{tabular}{|l|r|l|r|l|r|}
\hline
thousand & $10^{3}$ & quattuordecillion & $10^{45}$ & octovigintillion & $10^{87}$ \\
\hline
million & $10^{6}$ & quindecillion & $10^{48}$ & novemvigintillion & $10^{90}$ \\
\hline
billion & $10^{9}$ & sexdecillion & $10^{51}$ & trigintillion & $10^{93}$ \\
\hline
trillion  & $10^{12}$ & septdecillion & $10^{54}$ & untrigintillion & $10^{96}$ \\
\hline
quadrillion  & $10^{15}$ & octodecillion & $10^{57}$ & duotrigintillion & $10^{99}$ \\
\hline
quintillion  & $10^{18}$ & novemdecillion & $10^{60}$ & \textbf{googol} & $10^{100}$ \\
\hline
sexillion    & $10^{21}$ & vigintillion & $10^{63}$ & & \\
\hline
septillion   & $10^{24}$ & unvigintillion & $10^{66}$ & & \\
\hline
octillion    & $10^{27}$ & duovigintillion & $10^{69}$ & & \\
\hline
noniliion  & $10^{30}$ & trevigintillion & $10^{72}$ & & \\
\hline
decillion  & $10^{33}$ & quattuorvigintillion & $10^{75}$ & & \\
\hline
undecillion   & $10^{36}$ & quinvigintillion & $10^{78}$ & & \\
\hline
duodecillion  & $10^{39}$ & sexvigintillion & $10^{81}$ & & \\
\hline
tredecillion  & $10^{42}$ & seprvigintillion & $10^{84}$ & & \\
\hline
\end{tabular}
\end{center}

表中最后一个大单位古格尔(googol)是在1920年由9岁的米尔顿$\cdot$西洛塔(Milton Sirotta)想出的名字。这个数字是1后面跟着100个零。著名的互联网公司谷歌的名字就来自它\cite{Wikipedia-Googol}。

\begin{wrapfigure}{L}{0.3\textwidth}
 \centering
 \includegraphics[scale=0.5]{img/Zeno.eps}
 \captionsetup{labelformat=empty}
 \caption{芝诺,约490BC - 425BC}
 \label{fig:Zeno-of-Elea}
\end{wrapfigure}

超越一切具体大数的无穷是否存在不仅是一个数学问题,还是一个哲学问题。无穷大还直接导致另一个概念——无穷小。古代中国的哲学家庄子在《天下篇》中说:“一尺之棰,日取其半,万世不竭。”,古希腊的哲学家,埃利亚学派的芝诺(Zeno of Elea)提出了著名的四个悖论,它们都与无穷有关。

\begin{figure}[htbp]
%\begin{wrapfigure}{R}{0.3\textwidth}
 \centering
 \includegraphics[scale=0.4]{img/Achilles-paradox.eps}
 %\captionsetup{labelformat=empty}
 \caption{阿基里斯与乌龟悖论}
 \label{fig:Achilles-paradox}
%\end{wrapfigure}
\end{figure}

第一个悖论最为人们所津津乐道。名叫阿基里斯与乌龟悖论。阿基里斯是荷马史诗《伊里亚特》中的英雄,以善跑著称。这个悖论说:如果让爬得很慢的乌龟在阿基里斯前面一段路程出发,那么阿基里斯将永远追不上乌龟。这时因为,阿基里斯为了赶上乌龟,必须先到达乌龟的出发点$A$,但当阿基里斯到达$A$点时,乌龟已经在这段时间前进到了$B$点。但当阿基里斯到达$B$点时,乌龟又已经到了前面的$C$点……以此类推,两者间的距离虽然越来越近,但阿基里斯永远落在乌龟的后面而追不上乌龟。如图\ref{fig:Achilles-paradox}所示。但是这与我们生活中的常识是不相符的。这个悖论的推理是如此让人信服,以至于千百年来吸引了无数学者的研究。刘易斯$\cdot$卡罗尔(Lewis Carrol)、侯世达(Douglas Hofstadter)甚至拿乌龟和阿基里斯作为文学作品中的主人公。

第二个悖论叫作“二分悖论”。这个悖论说,如果阿基里斯想从$A$到$B$,那么他必须先走到$1/2$的位置。同样在此之前,他必须要到达$1/4$的位置。而为了到达这一位置,他必须先到达$1/8$的位置……以此类推。由于这样的中点有无限多个,阿基里斯永远也也无法到达目的。芝诺的这个悖论实际上说明了运动根本无法发生。

\begin{figure}[htbp]
%\begin{wrapfigure}{R}{0.3\textwidth}
 \centering
 \includegraphics[scale=0.4]{img/Dichotomy-paradox.eps}
 %\captionsetup{labelformat=empty}
 \caption{二分悖论}
 \label{fig:Dichotomy-paradox}
%\end{wrapfigure}
\end{figure}

第三个悖论叫作“飞矢不动悖论”,它从另一个角度描述无穷导致运动无法发生。芝诺指出,任何物体待在相同的位置都不叫运动,可是飞行的箭矢在任一时刻不也是待在一个地方么?这样说来,自然飞失也是不动的。如果说前两个悖论是由于分割空间导致的,则这个悖论是由对时间的分割导致的。

\begin{figure}[htbp]
%\begin{wrapfigure}{R}{0.3\textwidth}
 \centering
 \includegraphics[scale=0.4]{img/Arrow-paradox.eps}
 %\captionsetup{labelformat=empty}
 \caption{飞失不动悖论}
 \label{fig:Arrow-paradox}
%\end{wrapfigure}
\end{figure}

第四个悖论叫作“运动场悖论”。这一悖论主要针对时间原子论的观点,即认为存在最小的不可分割的时间单位。如图\ref{fig:Moving-rows-paradox}所示,运动场中有3列人。最初他们都收尾对齐。在最小的时间单元内,A列不动,B向右移一个单位,而$\Gamma$向左移动一个单位。容易得知,相对B而言,$\Gamma$其实移动列两个单位。这就意味着,应该存在这一让$\Gamma$相对于B移动一个单位的时间。而这一时间应该是最小单位时间的一半。但如果存在不可分割的“时间原子”,那么这两个时间就是相同的,即最小时间和它的一半相等。

芝诺悖论并不复杂,稍加琢磨就能理解。但是导出的结果却出人意料。根据生活中的常识,运动和时间是如此真实,阿基里斯不可能赶不上乌龟。可是驳倒这些悖论却并不容易,从亚里士多德到罗素,从阿基米德到赫尔曼$\cdot$外尔,都对芝诺悖论提出了各种不同的解法\cite{Wikipedia-Zeno}。

\begin{figure}[htbp]
%\begin{wrapfigure}{R}{0.3\textwidth}
 \centering
 \includegraphics[scale=0.3]{img/Moving-rows-paradox.eps}
 %\captionsetup{labelformat=empty}
 \caption{运动场悖论}
 \label{fig:Moving-rows-paradox}
%\end{wrapfigure}
\end{figure}

芝诺(约公元前490年——公元前425年),古希腊哲学家,生于意大利半岛南部的埃利亚。所以我们常称其为埃利亚的芝诺(Zeno of Elea)。关于他的生平,缺少可靠的文字记载。据传,他早年是一个自学成才的乡村孩子,一生经历坎坷,最终遭到一位暴君的陷害而被拘捕、拷打、直至被处死\cite{HanXueTao16}。

芝诺是继毕达哥拉斯学派之后在意大利新出现的一个哲学学派——埃利亚学派的代表人物之一,这一学派的领袖是芝诺的老师巴门尼德。巴门尼德认为整个世界是个不变的整体,即“不变的一”,运动、变化与多样性都只是幻象。

芝诺以其悖论闻名。他一生曾巧妙地构想出40多个悖论,在流传下来的悖论中以关于运动的四个“无限微妙、无限深邃”的悖论最为著名。他提出这些悖论很可能是为他老师的哲学观点辩护。

芝诺悖论在当时曾给古希腊人造成深深的困惑。而芝诺悖论所涉及的对时间、空间、无限、连续、运动的看法,也都在极长的历史岁月中困扰着后来的哲学家和数学家。如何了解和认识无穷,如何使用无穷成为了摆在古希腊人面前必须解决的问题。

亚里士多德对芝诺悖论进行了深入的思考(我们今天对芝诺悖论的了解,其实是来自亚里士多德的著作《物理学》),并做出一项对后世数学发展具有深远影响的工作。他把无穷的概念区分为实无穷和潜无穷。所谓潜无穷或潜无限,是指无限在永远延伸着,是一种变化的、不断产生出来概念。它永远在构造中,永远完成不了,是潜在的,而不是实在的。自然数就是一种潜无穷,对于任何一个自然数,我们都可以找到它的后继,也就是一个更大的自然数。欧几里得几何中的直线也是一种潜无穷,我们可以按需延伸直线\footnote{欧几里得避免使用“无限延伸”这样的说法,而代之以按需任意延伸。这在古希腊是常见的一种处理。}。所谓实无穷是指把无限的整体本身作为一个实在的单位,是已经构造出来的东西。也就是把无限对象看作可完成的过程或无穷整体。

在做了这种区分后,亚里士多德承认存在潜无穷,但是拒绝承认实无穷的概念。他对实无穷的排斥深刻而长远地影响了日后数学的发展\cite{HanXueTao16}。亚里士多德代表了当时古希腊的哲学观点,关于潜无穷和实无穷的概念区分以及争论一直影响至今。尽管对无穷的概念仍然心存疑虑,古希腊的数学家们借助潜无穷的思想取得了令人赞叹的成就。其中之一就是欧几里得证明了存在无穷多的素数。这一证明被人们认为是历史上最优美的证明之一。

\begin{theorem}
《几何原本》第九卷,命题20。预先给定任意多的素数,则有比它们更多的素数\cite{Elements}。
\end{theorem}

欧几里得在叙述这个命题时,小心谨慎地避免使用无穷这样的说法,通观《几何原本》一书这样的例子还有很多。欧几里得在证明这个命题时,使用了著名的反证法。我们用现代的语言来描述这一证明。

\begin{proof}
假设只存在有限多个素数,$p_1, p_2, ..., p_n$。我们构造一个新数:
\[
p_1 p_2 ... p_n + 1
\]
也就是把这$n$个素数乘起来再加一。这个数要么是素数,要么不是素数。

\begin{itemize}
\item 如果它是素数,明显这个数不等于$p1$到$p_n$中的任何一个,这就在有限多个素数中又增加了一个新的素数;
\item 如果这个数不是素数,那么它就存在一个素因子$p$。但是由于$p1$到$p_n$中的任何一个都不能整除我们构造的这个数,所以素数$p$与任何$p_1$到$p_n$中的数都不同,是一个新的素数。
\end{itemize}
所以在任何情况下,我们都可以获得一个新的素数。这与有限个素数的假设矛盾,因此存在无穷多的素数。
\end{proof}

欧几里得用反正法得到了一种“存在性证明”,他证明了存在无穷多的素数,但却没有给出怎样得到这些素数。这在我们今天看来,是很自然的一种处理。然而在十九世纪末二十世纪初却引发了数学根本性的争论,我们在下一章将详细讲述这一内容。

阿基米德的穷竭法

实用观点的发展,微积分,无穷的符号∞

潜无穷的例子,流与惰性求值

流的范畴论解释

实无穷的思考

希尔伯特旅馆

一一对应与无穷集合

康托尔与戴得金

利用(可数)无穷定义斐波那契数列和哈明数列

可数无穷和不可数无穷——实数集

对角线证明

戴得金分割

连续统假设

无穷与艺术

非欧几何

\begin{wrapfigure}{L}{0.4\textwidth}
 \centering
 \includegraphics[scale=1.0]{img/circle-limit-III-1959.eps}
 \captionsetup{labelformat=empty}
 \caption{埃舍尔《圆极限$\cdot$3》1959}
 \label{fig:Penrose-triangle}
\end{wrapfigure}

% Mathematicians aren't satisfied because they know there are no solutions up to four million or four billion, they really want to know that there are no solutions up to infinity. -- Andrew Wiles

\ifx\wholebook\relax \else
\begin{thebibliography}{99}

\bibitem{De-linfini-2018}
[法] 让-皮埃尔$\cdot$卢米涅,马克$\cdot$拉雪茨-雷 著,孙展 译. 从无穷开始——科学的困惑与疆界. 人民邮电出版社. 2018. ISBN: 9787115479198

\bibitem{Noguchi2007}
[日] 野口哲也 著,刘慧 韩丽红 译. 数学原来可以这样学. 湖南人民出版社. 2014. ISBN: 9787556100897
% Tetsunori Noguchi. SUGAKUTEKI SENSE GA MINITUKU RENSHUCHO.

\bibitem{Wikipedia-Googol}
Wikipedia. ``Googol''. \url{https://en.wikipedia.org/wiki/Googol}

\bibitem{Wikipedia-Zeno}
Wikipedia. ``Zeno's Paradoxes''. \url{https://en.wikipedia.org/wiki/Zeno's_paradoxes}

\bibitem{HanXueTao16}
韩雪涛 ``数学悖论与三次数学危机''. 人民邮电出版社. 2016, ISBN: 9787115430434

\bibitem{Elements}
[古希腊] 欧几里得 著,兰纪正 朱恩宽 译,梁宗巨 张毓新 徐伯谦 校订 ``几何原本''. 译林出版社. 2014, ISBN: 9787544750066

\end{thebibliography}

\expandafter\enddocument
%\end{document}

\fi
