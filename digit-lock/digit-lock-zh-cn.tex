\ifx\wholebook\relax \else

\documentclass[UTF8]{article}

\input{../common-zh-cn.tex}

\setcounter{page}{1}

\begin{document}

% ================================================================
%                 Digit lock
% ================================================================

\title{锁车趣题}

\author{刘新宇
\thanks{{\bfseries 刘新宇} \newline
  Email: liuxinyu95@gmail.com \newline}
  }

\maketitle
\fi

\markboth{锁车趣题}{Unplugged}

\section{共享自行车}

在我撰写这一章节的时候,共享自行车正在流行。一些公司在城市的大街小巷投放一些自行车。如果一个人要使用,他需要支付一定的押金,然后就可以打开车锁骑走,共享自行车公司按照使用时间收费。到达目的地后,从押金中扣除使用费就可以了。

为了实现在任何时间、地点,都能够租、还车,自行车公司通过使用当时已经普及的智能手机,来实现自行车的租用、计时、收费。一个典型的场景大致如下:

\begin{enumerate}
\item 一个行人$p$在路边看到一辆上锁的共享自行车,这辆自行车上有一个唯一的编号$x$,他通过自己的手机联系自行车公司,告知他要使用这辆自行车;
\item 自行车公司收到这一行人$p$的请求,向他收取使用自行车$x$的押金;
\item 行人通过自己的智能手机向自行车公司支付押金;
\item 自行车公司收到押金后,通过某种方式打开自行车$x$的车锁;
\item 行人将自行车骑到某地点后,将自行车停好,并通过手机告知自行车公司他要还车;
\item 自行车公司收到请求后,根据自行车x的使用时间,从押金中扣除费用,将押金退还行人$p$,同时通过某种方式将自行车$x$锁好。
\end{enumerate}

这一使用方式中有两个环节非常关键。

%% \begin{figure}[htbp]
%%   \centering
%%   \includegraphics[scale=1]{img/divide-by-m.ps}
%%   \caption{数组划分的过程。所有位于$0 \leq i < left$的元素满足$x[i] \leq m$,所有位于$left \leq i < right$的元素满足$x[i] > m$,剩余的元素尚未处理。} \label{fig:divide}
%% \end{figure}


% ================================================================
%                 Short summary
% ================================================================
\section{小结}

\ifx\wholebook\relax \else
%% \begin{thebibliography}{99}

%% \bibitem{fp-pearls}
%% Richard Bird. ``Pearls of functional algorithm design''. Cambridge University Press; 1 edition (November 1, 2010). ISBN-10: 0521513383

%% \bibitem{Bentley}
%% Jon Bentley. ``Programming Pearls(2nd Edition)''. Addison-Wesley Professional; 2 edition (October 7, 1999). ISBN-13: 978-0201657883 (中文版:《编程珠玑》)

%% \bibitem{okasaki-book}
%% Chris Okasaki. ``Purely Functional Data Structures''. Cambridge university press, (July 1, 1999), ISBN-13: 978-0521663502

%% \bibitem{CLRS}
%% Thomas H. Cormen, Charles E. Leiserson, Ronald L. Rivest and Clifford Stein. ``Introduction to Algorithms, Second Edition''. The MIT Press, 2001. ISBN: 0262032937. (中文版:《算法导论》)

%% \end{thebibliography}

\expandafter\enddocument
%\end{document}

\fi
