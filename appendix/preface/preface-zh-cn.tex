\ifx\wholebook\relax \else

\documentclass[UTF8]{article}

\input{../../common-zh-cn.tex}

\setcounter{page}{1}

\begin{document}

\title{前言}

\author{刘新宇
\thanks{{\bfseries 刘新宇} \newline
  Email: liuxinyu95@gmail.com \newline}
  }

\maketitle
\fi

\markboth{前言}{编程中的数学}

\chapter*{前言}
%\phantomsection  % so hyperref creates bookmarks
%\addcontentsline{toc}{chapter}{前言}

我讲一个从马爷爷那里听来的故事。有一年春节的时候,北京地坛公园的庙会里人山人海,小朋友们拿着压岁钱在各种摊位上买自己喜欢的玩具。
有一个摊位上围了一群人。地上一字排开摆了九个小玩具,每个玩具上依次贴着一元、二元、三元……一直到九元的标签。摊主一边向大家吆喝,
一边讲解游戏规则:“大家快来玩套圈游戏!一次一元钱,你扔一个,我扔一个,一个玩具只能套一个圈。谁有三个圈套中的玩具加到一起值
十五元,就可以把套中的玩具都赢走。”

\vspace{5mm}
\begin{tabular}{|c|c|c|c|c|c|c|c|c|}
\hline
1 & 2 & 3 & 4 & 5 & 6 & 7 & 8 & 9 \\
\hline
\end{tabular}
\vspace{5mm}

有个小男孩掏出一元钱,拿了一个红色的圈,使劲一扔。真准!正好套在了七块钱的玩具汽车上,摊主拿出一个蓝色的圈,一下子套中了八块钱
的兔爷。

\vspace{5mm}
\begin{tabular}{|c|c|c|c|c|c|c|c|c|}
\hline
1 & 2 & 3 & 4 & 5 & 6 & 7 & 8 & 9 \\
\hline
  &   &   &   &   &   & 男孩 & 摊主 & \\
\hline
\end{tabular}
\vspace{5mm}

小男孩又花了一元钱,这次他套中了价值两元的一个风车,这样只要他下次再套中那个六元的九连环,他就赢了。可这次摊主不慌不忙地
把蓝圈准准地套住了那只九连环。

\vspace{5mm}
\begin{tabular}{|c|c|c|c|c|c|c|c|c|}
\hline
1 & 2 & 3 & 4 & 5 & 6 & 7 & 8 & 9 \\
\hline
  & 男孩  &   &   &   & 摊主  & 男孩 & 摊主 & \\
\hline
\end{tabular}
\vspace{5mm}

这下可糟了,如果接下来摊主再套中那个一元钱的中国结,小男孩就要输了。小男孩涨红了脸,只能抢先去套那个中国结,他试了两次
终于套中了。

\vspace{5mm}
\begin{tabular}{|c|c|c|c|c|c|c|c|c|}
\hline
1 & 2 & 3 & 4 & 5 & 6 & 7 & 8 & 9 \\
\hline
男孩  & 男孩  &   &   &   & 摊主  & 男孩 & 摊主 & \\
\hline
\end{tabular}
\vspace{5mm}

摊主接下来扔了一个圈,套住了四元钱的孙悟空面具。小男孩套住了五元钱的小猪存钱罐。

\vspace{5mm}
\begin{tabular}{|c|c|c|c|c|c|c|c|c|}
\hline
1 & 2 & 3 & 4 & 5 & 6 & 7 & 8 & 9 \\
\hline
男孩  & 男孩  &   & 摊主  & 男孩  & 摊主  & 男孩 & 摊主 & \\
\hline
\end{tabular}
\vspace{5mm}

但是摊主扔出一个圈,套住了三元的不倒翁。由于3 + 4 + 8 = 15,小男孩输了。

\vspace{5mm}
\begin{tabular}{|c|c|c|c|c|c|c|c|c|}
\hline
1 & 2 & 3 & 4 & 5 & 6 & 7 & 8 & 9 \\
\hline
男孩  & 男孩  & 摊主  & 摊主  & 男孩  & 摊主  & 男孩 & 摊主 & \\
\hline
\end{tabular}
\vspace{5mm}

周围的人看着很好玩,也纷纷掏钱套圈玩。马爷爷看了一阵,觉得很奇怪,大多数
情况下都是摊主赢,偶尔会平局。马爷爷怀疑摊主一定有什么秘密,他只是为了避免
人们怀疑,有时才故意输掉游戏。

马爷爷回到家,电视里正在讲中国古老的《河图洛书》,据说在文字尚未发明之前,伏羲
治理天下的时候,在黄河支流,有乌龟背负着神秘的图案。如果把图案中的圆点数目用
现代的方法表示出来,就是一个数学上的三阶幻方。

%\begin{wrapfigure}{R}{0.3\textwidth}
\begin{figure}[htbp]
 \centering
 \subcaptionbox{洛书}[0.45\linewidth]{ \includegraphics[scale=0.8]{img/luo-shu.png}}
 \subcaptionbox{三阶幻方}[0.45\linewidth]{
   \begin{tabular}{|c|c|c|}
   \hline
   4 & 9 & 2 \\
   \hline
   3 & 5 & 7 \\
   \hline
   8 & 1 & 6 \\
   \hline
   \end{tabular}
   \vspace{8mm}
 }
 \captionsetup{labelformat=empty}
 \caption{}
 \label{fig:luo-shu}
\end{figure}
%\end{wrapfigure}

\ifx\wholebook\relax \else

\expandafter\enddocument
%\end{document}

\fi
