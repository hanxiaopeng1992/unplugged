\ifx\wholebook\relax \else

\documentclass[UTF8]{article}

\input{../../common-zh-cn.tex}

\setcounter{page}{1}

\begin{document}

% ================================================================
%                 Digit lock
% ================================================================

\title{自然数}

\author{刘新宇
\thanks{{\bfseries 刘新宇} \newline
  Email: liuxinyu95@gmail.com \newline}
  }

\maketitle
\fi

\markboth{自然数}{计算机程序的代数结构}

\section{加法交换律的证明}

为了证明加法的交换律$a + b = b + a$。我们首先证明三个结论。第一个是说对于任何自然数都有:

\be
0 + a = a
\label{eq:left-zero}
\ee

也就是说,加法左侧的零可以消去。首先看起始情况,当$a=0$时,根据加法定义的第一条规则有:

\[
0 + 0 = 0
\]

其次是递推情况,设$0 + a = a$,我们要推出$0 + a' = a'$。

\[
\begin{array}{rlr}
0 + a' & = (0 + a)' & \text{加法定义的规则二} \\
       & = a' & \text{递推假设}
\end{array}
\]

接下来我们定义0的后继为1,并证明第二个重要结论:

\be
a' = a + 1
\label{eq:one-succ}
\ee

这是因为:

\[
\begin{array}{rlr}
a' & = (a + 0)' & \text{加法定义的规则一} \\
   & = a + 0' & \text{加法定义规则二} \\
   & = a + 1
\end{array}
\]

我们需要补充定义0的后继为1,然后就可以利用归纳公理进行证明。
第二个要证明的结论是交换律的一个特例:

\be
$a + 1 = 1 + a$。
\label{eq:one-commu}
\ee

首先看$a = 0$的起始情况。

\[
\begin{array}{rlr}
0 + 1 & = 1 & \text{刚证明的加法左侧零可消去} \\
      & = 1 + 0 & \text{加法定义的第一条规则}
\end{array}
\]

然后是递推情况,设$a + 1 = 1 + a$成立,我们要推出$a' + 1 = 1 + a'$。

\[
\begin{array}{rlr}
a' + 1 & = a' + 0' & \text{1是0的后继} \\
       & = (a' + 0)' & \text{加法定义的第一条规则}
\end{array}
\]


如果我们定义1是0的后继,则任何数$n$的后继可以写成:

\be
$n' = n + 1$
\label{eq:one-succ}
\ee

这是因为:

\[
\begin{array}{rlr}
n' & = (n + 0)' & \text{根据加法定义的第一条规则} \\
   & = n + 0' & \text{根据加法定义的第二条规则} \\
   & = n + 1 & \text {定义0的后继为1} \\
\end{array}
\]

有了这两个结论,我们就可以着手证明加法交换律了。我们首先证明$b=0$时交换律成立。根据加法定义的第一条,我们有$a + 0 = a$;同时根据刚才证明的结论一,又有$0 + a = a$。这就证明了$a + 0 = 0 + a$。
然后我们证明递推情况。假设$a + b = b + a$成立,我们要推出$a + b' = b' + a$。

\[
\begin{array}{rlr}
a + b' & = (a + b)' & \text{根据加法定义的第二条规则} \\
       & = (b + a)' & \text{递推假设} \\
       & = b + a' & \text{加法定义的第二条规则} \\
       & = b + a + 1 & \text{根据刚证明的结论二,即(\ref{eq:one-succ})} \\
       & = b + 1 + a & \text{} \\
       & = (b + 1) + a & \text{加法结合律} \\
       & = b' + a &
\end{array}
\]


而$b + a$

\begin{Exercise}
\begin{enumerate}
\item 证明对于任何自然数都有$0 + a = a$
\item 定义0的后继为1,证明对于任何自然数都有$a \cdot 1 = a$
\item 证明乘法结合律和交换律
\item 证明乘法分配律
\end{enumerate}
\end{Exercise}

\section{自然数的代数结构}
具体例子
群,monoid和半群的概念

\section{自然数程序}
使用自然数的代数结构的程序的例子

将一类程序的结构抽象为代数结构

使用自然数的代数结构抽象更多的程序

\ifx\wholebook\relax \else
\begin{thebibliography}{99}

\bibitem{wiki-number}
Wikipedia. ``古代计数系统的历史''. \url{https://en.wikipedia.org/wiki/History_of_ancient_numeral_systems}

\bibitem{trip-to-number-kindom}
[美]卡尔文$\cdot$C$\cdot$克劳森. ``数学旅行家:漫游数王国''. 袁向东、袁钧译,上海教育出版社。ISBN:

\bibitem{wiki-babylonian-num}
Wikipedia. ``古巴比伦数字''. \url{https://en.wikipedia.org/wiki/Babylonian_numerals}
\end{thebibliography}

\expandafter\enddocument
%\end{document}

\fi
