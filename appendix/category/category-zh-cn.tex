\ifx\wholebook\relax \else

\documentclass[UTF8]{article}

\input{../../common-zh-cn.tex}

\setcounter{page}{1}

\begin{document}

\title{范畴}

\author{刘新宇
\thanks{{\bfseries 刘新宇} \newline
  Email: liuxinyu95@gmail.com \newline}
  }

\maketitle
\fi

\markboth{范畴}{编程中的数学}

\section{积和余积的唯一性}

积和余积是唯一的么?下面的定理回答了这个问题:

\begin{lemma}
对于范畴$\pmb{C}$中的一对对象$A$和$B$,令下图中的对象和箭头

\begin{center}
\begin{tikzpicture}
  \matrix (m) [matrix of math nodes,
               row sep=2em, column sep=2em, minimum width=2em]{
    P &   &  &   & I \\
      & A &  & A &   \\
      & B &  & B &   \\
    Q &   &  &   & J \\};
  \path[-stealth]
    % left
    (m-1-1) edge node [above] {$p_A$} (m-2-2)
    (m-1-1) edge node [below] {$p_B$} (m-3-2)
    (m-4-1) edge node [above] {$q_A$} (m-2-2)
    (m-4-1) edge node [below] {$q_B$} (m-3-2)
    % right
    (m-2-4) edge node [above] {$i_A$} (m-1-5)
    (m-2-4) edge node [above] {$j_A$} (m-4-5)
    (m-3-4) edge node [below] {$i_B$} (m-1-5)
    (m-3-4) edge node [below] {$j_B$} (m-4-5);
\end{tikzpicture}
\end{center}

各自为一对

\begin{center}
积 \quad \quad \quad 余积
\end{center}

的楔形。则

\[
P, Q \quad \quad \quad I, J
\]

是同构的楔形,存在唯一的箭头:

\begin{center}
\begin{tikzpicture}
  \matrix (m) [matrix of math nodes,
               row sep=2em, column sep=3em, minimum width=2em]{
    P & & I \\
    Q & & J \\};
  \path[-stealth]
    % left
    (m-1-1.south west) edge node [left] {$f$} (m-2-1.north west)
    (m-2-1.north east) edge node [right] {$g$} (m-1-1.south east)
    % right
    (m-1-3.south west) edge node [left] {$f$} (m-2-3.north west)
    (m-2-3.north east) edge node [right] {$g$} (m-1-3.south east);
\end{tikzpicture}
\end{center}

使得:

\[
\begin{array}{lcl}
  \begin{cases}
    p_A = q_A \circ f & p_B = q_B \circ f \\
    q_A = p_A \circ g & q_B = p_B \circ g
  \end{cases}
  & \quad &
  \begin{cases}
    i_A = g \circ j_A & i_B = g \circ j_B \\
    j_A = f \circ i_A & j_B = f \circ i_B
  \end{cases}
\end{array}
\]

并且,$f$和$g$是互逆的同构箭头对。
\end{lemma}

\begin{proof}
我们只证明左侧积的部分,右侧余积部分的证明与此类似。给定$A, B$,对象$Q$和两个箭头$q_A, q_B$构成的楔形是一个积。我们把对象$P$和一对箭头$p_A, p_B$看成是另一个楔形。根据积的定义,存在唯一的媒介箭头$f$满足:

\[
p_A = q_A \circ f \quad \text{和} \quad p_B = q_B \circ f
\]

交换$P$和$Q$(另$P$为积,$Q$为任意楔形),又可以得到:

\[
q_A = p_A \circ g \quad \text{和} \quad q_B = p_B \circ g
\]

这样就有:

\[
\begin{cases}
p_A \circ g \circ f = q_A \circ f = p_A \\
p_B \circ g \circ f = q_B \circ f = p_B
\end{cases}
\]

以及:

\[
\begin{cases}
q_A \circ f \circ g = p_A \circ g = q_A \\
q_B \circ f \circ g = p_B \circ g = q_B \\
\end{cases}
\]

因此:

\[
g \circ f = id_P \quad \quad \quad f \circ g = id_Q
\]

\end{proof}

这一结论说明,如果两个对象存在积(或者余积)则它是唯一的。但是积和余积并不总是同时存在,甚至有可能都不存在。

\ifx\wholebook\relax \else

\expandafter\enddocument
%\end{document}

\fi
