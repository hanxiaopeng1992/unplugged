\ifx\wholebook\relax \else

\documentclass[UTF8]{article}

\input{../common-zh-cn.tex}

\setcounter{page}{1}

\begin{document}

% ================================================================
%                 Digit lock
% ================================================================

\title{自然数}

\author{刘新宇
\thanks{{\bfseries 刘新宇} \newline
  Email: liuxinyu95@gmail.com \newline}
  }

\maketitle
\fi

\markboth{自然数}{计算机程序的代数结构}

\section{数的诞生}
数的概念是伴随着我们人类进化产生的。有人认为,数直接催生了人类的语言和文字。我们的祖先在长期的狩猎——采集活动中,逐渐掌握了数的概念。
最早可能是简单的数数,例如数清果实的数量。随着文明的发展,人们逐渐开始进行物品交易。随着交易物品数量增长,很快需要采用助记工具来处理更大的数。
考古发现在今天的伊朗一代,人们在公元前4000年开始使用陶制的小球来辅助计数。比如用两个刻有十字的小球代表两只羊,同时还有代表十只羊,二十只羊的不同小球。
为了防止忘记或者篡改数过的数目,人们还把这些小球放在陶罐中用泥土封存起来。图\ref{fig:clay-token}是乌鲁克(Uruk)时期的陶制计数罐和小球\cite{wiki-number}。
交易过程中,人们可以通过这些工具掌握货物的数量\cite{trip-to-number-kindom}。

\begin{figure}[htbp]
 \centering
 \includegraphics[scale=0.5]{img/clay-envelope.eps}
 \caption{卢浮宫陈列的乌鲁尔时期的计数陶罐和一组计数陶球。}
 \label{fig:clay-token}
\end{figure}

随着交易的增加和数目的增大,这样的陶罐和小球就不够方便了。大约公元前3500年,美索不达米亚的苏美尔人开始在泥板上刻化符号来纪录交易。将泥板烤硬后就可以方便的保存纪录。人们在这一时期用坚硬的笔在泥板上刻出不同的符号,同时表示交易的物品和数量。比如用一个象形符号表示五头牛,而用另一个象形符号表示十只羊。

更大进步发生在公元前3100年,从出土的泥板中,我们发现苏美尔人开始将数字从它代表的物品中抽象出来。人们不再使用一个符号同时表示物品和数量,而是用一个符号表示交易的数量,接下来用另一个符号表示交易的物品。例如先用一个符号表示五,然后跟上一个牛的象形符号表示五头牛;而表示五只羊的时候,人们用同样的符号表示五,然后再跟上一个羊的符号。这些泥板上的符号逐渐演变成了古巴比伦的楔形文字。

\begin{figure}[htbp]
 \centering
 \includegraphics[scale=0.8]{img/Babylonian_numerals.eps}
 \caption{巴比伦楔形文字中的数字\cite{wiki-babylonian-num}。}
 \label{fig:babylonian-num}
\end{figure}

抽象的数是智慧生命思维的产物。人们发现三个鸡蛋、三棵树、三个陶罐都可以用数字三来表示。这是一种强大的工具。从此,我们可以对抽象的数字进行操作,然后再把结果应用到各种具体事物上。例如我们可以把抽象的数字三加上一得到四,从而知道捡拾三个鸡蛋后再捡拾到一个鸡蛋会得到四个鸡蛋。同时我们也知道烧制三个陶罐后再烧制一个陶罐会得到四个陶罐。人们逐渐从解决单一问题发展到解决一类问题。

\begin{figure}[htbp]
 \centering
 \includegraphics[scale=0.2]{img/abstract-num.eps}
 \caption{具体的三个事物和抽象的数字三。}
 \label{fig:abstract-num}
\end{figure}

从数数开始,我们的老祖先逐渐发展出了操作抽象数字的方法,包括数字的加法、减法,更为强大的乘法,以及用于分配事物的除法。在丈量分割土地,计算谷物容量时,又逐渐将抽象的数和几何量联系起来。各个文明几乎分别独立的发现了数与形的内在规律。我们发现古埃及,古希腊,古中国都各自发现了毕达哥拉斯定理(勾股定理),古埃及人把它用于金字塔建造这样的伟大工程实践。从现代文明追根溯源,我们可以说自然数是数学和自然科学这条长河的源头。德国数学家克罗内克(Kronecker)说“上帝创造了自然数,其余都是人的工作。”\footnote{一说为整数,我们会在后继关于康托尔和无穷的章节再次提到克罗内克。}

\section{皮亚诺自然数公理}
古希腊的欧几里得在他的伟大著作《几何原本》中开创了公理化的方法。他用五条公理和五条公设作为基石,精心构建一条一条定理的证明。每一个结论都仅仅使用公理和此前已经证明的定理。最终构建出了叹为观止的几何大厦。然而对于自然数,长期以来人们却没有建立起它的公理化形式系统。也许人们一直认为自然数的结论是直观和显而易见的。直到1889年,意大利数学家皮亚诺\footnote{皮亚诺不仅是数学家和逻辑学家,还是语言学家。他是数理逻辑和集合理论的先驱。他直接影响了罗素和怀特海的著作《数学原理》以及此后法国布尔巴基学派的纲领。他还创立了国际语(又称为“无屈折拉丁语”的人工语言)。}才为自然数建立起了严格的公理化系统。这就是著名的皮亚诺公里(Peano Axioms)。也许是上帝的巧合,欧几里得几何公理有五条,皮亚诺公理也有五条:

\begin{figure}[htbp]
 \centering
 \includegraphics[scale=0.2]{img/Peano.eps}
 \caption{朱塞佩$\cdot$皮亚诺(Giuseppe Peano)1858 - 1932。}
 \label{fig:abstract-num}
\end{figure}

\begin{enumerate}
\item 0是自然数。用符号表示为$\exists 0 \in N$;
\item 每个自然数都有它的下一个自然数,称为它的后继。用符号表示为$\forall n \in N, \exists n' = succ(n) \in N$;
\end{enumerate}

似乎仅仅有这两条公理,我们已经能够定义出无穷无尽的自然数了,从0开始,下一个是1,下一个是2,接下来是3,……,接下来是某个$n$,下一个是$n+1$,……以至无穷。但是好挑刺的数学家给出了一个反例:考虑只有两个元素$\{0, 1\}$组成的数字系统,定义1的后继为0,0的后继为1。这样也满足上面的两条公理,却不是我们想像中的自然数

为此我们还需要第三条皮亚诺公理来排出这种情况。

\begin{enumerate}
  \setcounter{enumi}{2}
  \item 0不是任何自然数的后继。用符号表示为$\forall n \in N: n' \neq 0$;
\end{enumerate}

仅仅有这三条公理就够了么?我们还可以给出一个反例:考虑有限元素$\{0, 1, 2\}$组成的数字系统,定义0的后继是1,1的后继是2,2的后继还是2。这样也能满足上述三条公理。为此我们还需要第四条皮亚诺公理。

\begin{enumerate}
  \setcounter{enumi}{3}
  \item 不同的自然数有不同的后继数。或者说,如果两个自然数的后继数相同,那么这两个自然数相等。用符号表示为$\forall n, m \in N: n' = m' \Rightarrow n = m$;
\end{enumerate}

但是,仅仅用这四条公理仍然不够,因为可以存在这样的反例:考虑集合$\{0, 0.5, 1, 1.5, 2, 2.5, ...\}$,定义0的后继是1、1的后继是2……,0.5的后继是1.5、1.5的后继是2.5……但0.5不是任何元素的后继。为了排除这样的含有“不可达”元素的反例,还需要最后一条皮亚诺公理。

\begin{enumerate}
  \setcounter{enumi}{4}
  \item 如果自然数的某个子集包含0,并且其中每个元素都有后继元素。那么这个子集就是全体自然数。用符号表示为$\forall S \subset N: (0 \in S \land \forall n \in S \Rightarrow n' \in S) \Rightarrow S = N$.
\end{enumerate}

为什么公理5可以排出掉上述的反例呢?我们考虑集合$\{0, 0.5, 1, 1.5, 2, 2.5, ...\}$的一个子集$\{0, 1, 2, ...\}$。它包含0,并且每个元素都有后继元素,但是它不等于原集合。因为1.5、2.5……都不在这个子集中。所以它不满足第五条公理。公理五还有另外一个响亮的名字——归纳公里(Axiom of induction),它可以这样等价地描述:

\begin{enumerate}
  \setcounter{enumi}{4}
  \item 任意关于自然数的命题,如果证明了它对自然数0是对的,又假定它对自然数$n$为真时,可以证明它对$n'$也真,那么命题对所有自然数都真。(这条公理保证了数学归纳法的正确性)
\end{enumerate}

以上就是完整的五条皮亚诺公理,用它们可以构建出一阶算术系统,也称为皮亚诺算术系统\footnote{也有人从1开始,而不是0开始计算自然数。在皮亚诺当年的著作中,五条公理的顺序与此不同,其中第五条归纳公理被写在第三的位置上。}。

\section{自然数和计算机程序}
现代的计算机系统和在其上构建的程序已经非常的复杂和宏伟了。人们并非是先建立了计算机程序的公理系统,然后逐渐演绎出这些成果的。而是先取得了应用的巨大成功,然后才逐渐将计算机科学的基石数学化、形式化、严谨化的。这种有趣的现象在人类历史上已经不是第一次了。牛顿和莱布尼茨在17世纪发展了微积分,然后在几代数学家的手里应用到了各种领域,包括流体力学,天文学等等。但是直到19世纪才由柯西将微积分的理论严格化。

我们也模仿一下这样的过程,看看如何根据皮亚诺公理,用计算机程序定义自然数。在一个没有0、1、2、……这些我们熟悉的数字的程序系统中,我们可以这样定义自然数:

\lstset{language=Haskell}
\begin{lstlisting}
data Nat = Zero | Succ Nat
\end{lstlisting}

这一定义说:自然数Nat是某种数据,一个自然数或者为零,或者是另一自然数的后继。这里符号“|”表示互斥的关系。它自然蕴含了零不是任何自然数后继这一公理。在这一定义下,我们可以进一步定义出自然数的加法。

\begin{lstlisting}
m + Zero = m
m + (Succ n) = Succ (m + n)
\end{lstlisting}

\section{自然数的代数结构}
具体例子
群,monoid和半群的概念

\section{自然数程序}
使用自然数的代数结构的程序的例子

将一类程序的结构抽象为代数结构

使用自然数的代数结构抽象更多的程序

\ifx\wholebook\relax \else
\begin{thebibliography}{99}

\bibitem{wiki-number}
Wikipedia. ``古代计数系统的历史''. \url{https://en.wikipedia.org/wiki/History_of_ancient_numeral_systems}

\bibitem{trip-to-number-kindom}
[美]卡尔文$\cdot$C$\cdot$克劳森. ``数学旅行家:漫游数王国''. 袁向东、袁钧译,上海教育出版社。ISBN:

\bibitem{wiki-babylonian-num}
Wikipedia. ``古巴比伦数字''. \url{https://en.wikipedia.org/wiki/Babylonian_numerals}
\end{thebibliography}

\expandafter\enddocument
%\end{document}

\fi
