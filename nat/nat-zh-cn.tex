\ifx\wholebook\relax \else

\documentclass[UTF8]{article}

\input{../common-zh-cn.tex}

\setcounter{page}{1}

\begin{document}

% ================================================================
%                 Digit lock
% ================================================================

\title{自然数}

\author{刘新宇
\thanks{{\bfseries 刘新宇} \newline
  Email: liuxinyu95@gmail.com \newline}
  }

\maketitle
\fi

\markboth{自然数}{计算机程序的代数结构}

\section{数的诞生}

\begin{wrapfigure}{R}{0.3\textwidth}
%\begin{figure}[htbp]
 \centering
 \includegraphics[scale=0.4]{img/clay-envelope.eps}
 \caption{卢浮宫陈列的乌鲁尔时期的计数陶罐和一组计数陶球。}
 \label{fig:clay-token}
%\end{figure}
\end{wrapfigure}

数的概念是伴随着我们人类进化产生的。有人认为,数直接催生了人类的语言和文字。我们的祖先在长期的狩猎——采集活动中,逐渐掌握了数的概念。
最早可能是简单的数数,例如数清果实的数量。随着文明的发展,人们逐渐开始进行物品交易。随着交易物品数量增长,很快需要采用助记工具来处理更大的数。
考古发现在今天的伊朗一代,人们在公元前4000年开始使用陶制的小球来辅助计数。比如用两个刻有十字的小球代表两只羊,同时还有代表十只羊,二十只羊的不同小球。
为了防止忘记或者篡改数过的数目,人们还把这些小球放在陶罐中用泥土封存起来。图\ref{fig:clay-token}是乌鲁克(Uruk)时期的陶制计数罐和小球\cite{wiki-number}。
交易过程中,人们可以通过这些工具掌握货物的数量\cite{trip-to-number-kindom}。

随着交易的增加和数目的增大,这样的陶罐和小球就不够方便了。大约公元前3500年,美索不达米亚的苏美尔人开始在泥板上刻化符号来纪录交易。将泥板烤硬后就可以方便的保存纪录。人们在这一时期用坚硬的笔在泥板上刻出不同的符号,同时表示交易的物品和数量。比如用一个象形符号表示五头牛,而用另一个象形符号表示十只羊。

更大进步发生在公元前3100年,从出土的泥板中,我们发现苏美尔人开始将数字从它代表的物品中抽象出来。人们不再使用一个符号同时表示物品和数量,而是用一个符号表示交易的数量,接下来用另一个符号表示交易的物品。例如先用一个符号表示五,然后跟上一个牛的象形符号表示五头牛;而表示五只羊的时候,人们用同样的符号表示五,然后再跟上一个羊的符号。这些泥板上的符号逐渐演变成了古巴比伦的楔形文字。

%\begin{wrapfigure}{R}{0.5\textwidth}
\begin{figure}[htbp]
 \centering
 \includegraphics[scale=0.6]{img/Babylonian_numerals.eps}
 \caption{巴比伦楔形文字中的数字\cite{wiki-babylonian-num}。}
 \label{fig:babylonian-num}
\end{figure}
%\end{wrapfigure}

抽象的数是智慧生命思维的产物。人们发现三个鸡蛋、三棵树、三个陶罐都可以用数字三来表示。这是一种强大的工具。从此,我们可以对抽象的数字进行操作,然后再把结果应用到各种具体事物上。例如我们可以把抽象的数字三加上一得到四,从而知道捡拾三个鸡蛋后再捡拾到一个鸡蛋会得到四个鸡蛋。同时我们也知道烧制三个陶罐后再烧制一个陶罐会得到四个陶罐。人们逐渐从解决单一问题发展到解决一类问题。

\begin{wrapfigure}{R}{0.5\textwidth}
%\begin{figure}[htbp]
 \centering
 \includegraphics[scale=0.2]{img/abstract-num.eps}
 \caption{具体的三个事物和抽象的数字三。}
 \label{fig:abstract-num}
%\end{figure}
\end{wrapfigure}

从数数开始,我们的老祖先逐渐发展出了操作抽象数字的方法,包括数字的加法、减法,更为强大的乘法,以及用于分配事物的除法。在丈量分割土地,计算谷物容量时,又逐渐将抽象的数和几何量联系起来。各个文明几乎分别独立的发现了数与形的内在规律。我们发现古埃及,古希腊,古中国都各自发现了毕达哥拉斯定理(勾股定理),古埃及人把它用于金字塔建造这样的伟大工程实践。从现代文明追根溯源,我们可以说自然数是数学和自然科学这条长河的源头。德国数学家克罗内克(Kronecker)说“上帝创造了自然数,其余都是人的工作。”\footnote{一说为整数,我们会在后继关于康托尔和无穷的章节再次提到克罗内克。}

\section{皮亚诺自然数公理}

%\begin{wrapfigure}{R}{0.4\textwidth}
\begin{figure}[htbp]
 \centering
 \includegraphics[scale=0.2]{img/Peano.eps}
 \caption{朱塞佩$\cdot$皮亚诺(Giuseppe Peano)1858 - 1932。}
 \label{fig:abstract-num}
\end{figure}
%\end{wrapfigure}

古希腊的欧几里得在他的伟大著作《几何原本》中开创了公理化的方法。他用五条公理和五条公设作为基石,精心构建一条一条定理的证明。每一个结论都仅仅使用公理和此前已经证明的定理。最终构建出了叹为观止的几何大厦。然而对于自然数,长期以来人们却没有建立起它的公理化形式系统。也许人们一直认为自然数的结论是直观和显而易见的。直到1889年,意大利数学家皮亚诺\footnote{皮亚诺不仅是数学家和逻辑学家,还是语言学家。他是数理逻辑和集合理论的先驱。他直接影响了罗素和怀特海的著作《数学原理》以及此后法国布尔巴基学派的纲领。他还创立了国际语(又称为“无屈折拉丁语”的人工语言)。}才为自然数建立起了严格的公理化系统。这就是著名的皮亚诺公里(Peano Axioms)。也许是上帝的巧合,欧几里得几何公理有五条,皮亚诺公理也有五条:


\begin{enumerate}
\item 0是自然数。用符号表示为$\exists 0 \in N$;
\item 每个自然数都有它的下一个自然数,称为它的后继。用符号表示为$\forall n \in N, \exists n' = succ(n) \in N$;
\end{enumerate}

似乎仅仅有这两条公理,我们已经能够定义出无穷无尽的自然数了,从0开始,下一个是1,下一个是2,接下来是3,……,接下来是某个$n$,下一个是$n+1$,……以至无穷。但是好挑刺的数学家给出了一个反例:考虑只有两个元素$\{0, 1\}$组成的数字系统,定义1的后继为0,0的后继为1。这样也满足上面的两条公理,却不是我们想像中的自然数

为此我们还需要第三条皮亚诺公理来排出这种情况。

\begin{enumerate}
  \setcounter{enumi}{2}
  \item 0不是任何自然数的后继。用符号表示为$\forall n \in N: n' \neq 0$;
\end{enumerate}

仅仅有这三条公理就够了么?我们还可以给出一个反例:考虑有限元素$\{0, 1, 2\}$组成的数字系统,定义0的后继是1,1的后继是2,2的后继还是2。这样也能满足上述三条公理。为此我们还需要第四条皮亚诺公理。

\begin{enumerate}
  \setcounter{enumi}{3}
  \item 不同的自然数有不同的后继数。或者说,如果两个自然数的后继数相同,那么这两个自然数相等。用符号表示为$\forall n, m \in N: n' = m' \Rightarrow n = m$;
\end{enumerate}

但是,仅仅用这四条公理仍然不够,因为可以存在这样的反例:考虑集合$\{0, 0.5, 1, 1.5, 2, 2.5, ...\}$,定义0的后继是1、1的后继是2……,0.5的后继是1.5、1.5的后继是2.5……但0.5不是任何元素的后继。为了排除这样的含有“不可达”元素的反例,还需要最后一条皮亚诺公理。

\begin{enumerate}
  \setcounter{enumi}{4}
  \item 如果自然数的某个子集包含0,并且其中每个元素都有后继元素。那么这个子集就是全体自然数。用符号表示为$\forall S \subset N: (0 \in S \land \forall n \in S \Rightarrow n' \in S) \Rightarrow S = N$.
\end{enumerate}

为什么公理5可以排出掉上述的反例呢?我们考虑集合$\{0, 0.5, 1, 1.5, 2, 2.5, ...\}$的一个子集$\{0, 1, 2, ...\}$。它包含0,并且每个元素都有后继元素,但是它不等于原集合。因为1.5、2.5……都不在这个子集中。所以它不满足第五条公理。公理五还有另外一个响亮的名字——归纳公里(Axiom of induction),它可以这样等价地描述:

\begin{enumerate}
  \setcounter{enumi}{4}
  \item 任意关于自然数的命题,如果证明了它对自然数0是对的,又假定它对自然数$n$为真时,可以证明它对$n'$也真,那么命题对所有自然数都真。(这条公理保证了数学归纳法的正确性)
\end{enumerate}

以上就是完整的五条皮亚诺公理,用它们可以构建出一阶算术系统,也称为皮亚诺算术系统\footnote{也有人从1开始,而不是0开始计算自然数。在皮亚诺当年的著作中,五条公理的顺序与此不同,其中第五条归纳公理被写在第三的位置上。}。

\section{自然数和计算机程序}
现代的计算机系统和在其上构建的程序已经非常的复杂和宏伟了。人们并非是先建立了计算机程序的公理系统,然后逐渐演绎出这些成果的。而是先取得了应用的巨大成功,然后才逐渐将计算机科学的基石数学化、形式化、严谨化的。这种有趣的现象在人类历史上已经不是第一次了。牛顿和莱布尼茨在17世纪发展了微积分,然后在几代数学家的手里应用到了各种领域,包括流体力学,天文学等等。但是直到19世纪才由柯西将微积分的理论严格化。

我们也模仿一下这样的过程,看看如何根据皮亚诺公理,用计算机程序定义自然数。在一个没有0、1、2、……这些我们熟悉的数字的程序系统中,我们可以这样定义自然数\footnote{本书使用一种理想的计算机语言,并在每一章节的最后给出真实计算机语言的参考代码。}:

\lstset{language=Haskell}
\begin{lstlisting}
data Nat = zero | succ Nat
\end{lstlisting}

%% \[
%% N \triangleq zero | succ(N)
%% \]

这一定义说:一个自然数或者为零,或者是另一自然数的后继。这里符号“|”表示互斥的关系。它自然蕴含了零不是任何自然数后继这一公理。在这一定义下,我们可以进一步定义出自然数的加法。

\begin{lstlisting}
a + zero = a
a + (succ b) = succ (a + b)
\end{lstlisting}

加法定义包含两部分。首先任何自然数和零相加等于它本身;并且某个自然数和另一个数的后继相加,等于这两个数相加的后继。写成数学符号为:

\be
\begin{array}{l}
a + 0 = a \\
a + b' = (a + b)'
\end{array}
\ee

我们来验证一下2+3。自然数2为succ(succ zero),而3为succ(succ(succ zero))。根据加法的定义2+3为:

\begin{lstlisting}
  succ(succ zero) + succ(succ(succ zero))
= succ(succ(succ zero) + succ(succ zero))
= succ(succ(succ(succ zero) + succ zero))
= succ(succ(succ(succ(succ zero) + zero)))
= succ(succ(succ(succ(succ zero))))
\end{lstlisting}

最终结果的确是零的五重后继,也就是5。从零开始一次一次的重复使用后继函数的记法很麻烦。如果要表示100,这种记法要写很多行并且容易出错。为此,我们用下面的简单记法表示自然数$n$:

\be
n = foldn(0, succ, n)
\ee

它表示从零开始,不断叠加使用succ函数$n$次。$foldn$函数可以具体实现如下:

\be
\begin{array}{l}
foldn(z, f, 0) = z \\
foldn(z, f, n') = f(foldn(z, f, n))
\end{array}
\label{eq:foldn}
\ee

$foldn$定义了在自然数上的一种操作,只要令$z$为$zero$,令$f$为$succ$就可以实现叠加后继若干次,从而获得某个特定的自然数。我们可以用前几个自然数验证一下:

\begin{lstlisting}
foldn(zero, succ, 0) = zero
foldn(zero, succ, 1) = succ(foldn(zero, succ, 0)) = succ zero
foldn(zero, succ, 2) = succ(foldn(zero, succ, 1)) = succ(succ zero)
...
\end{lstlisting}

定义好加法之后,我们再来定义自然数的乘法:

\begin{lstlisting}
a . zero = zero
a . (succ b) = a . b + a
\end{lstlisting}

这里我们使用了刚才定义好的加法。这一定义写成数学符号为:

\be
\begin{array}{l}
a \cdot 0 = 0 \\
a \cdot b' = a \cdot b + a
\end{array}
\ee

\begin{wrapfigure}{R}{0.4\textwidth}
\centering
\begin{tikzpicture}[scale=0.8]
\filldraw[pattern=north west lines] (0, 0) rectangle (2, 1)
    (2, 0) rectangle (3, 1);
\draw (3, 0) rectangle (4.5, 1);
\draw (0, -1) rectangle (2, -2);
\filldraw[pattern=north west lines] (2, -1) rectangle (3, -2)
    (3, -1) rectangle (4.5, -2);
\end{tikzpicture}
\caption{加法结合律的几何证明。上下面积相等}
\end{wrapfigure}

与通常的观念不同,加法和乘法的交换律、结合律既不是公理,也不是公设。它们都是可以用皮亚诺公理和加法的定义严格证明的定理。我们来看看证明加法结合律的例子。加法结合律是说$(a + b) + c= a + (b + c)$。我们先证明当$c=0$时它是对的。根据加法定义的第一条规则:

\[
\begin{array}{rl}
(a + b) + 0 & = a + b \\
            & = a + (b + 0)
\end{array}
\]

然后是递推步骤,假设$(a + b) + c = a + (b + c)$成立,我们要推出$(a + b) + c' = a + (b + c')$。

\[
\begin{array}{rlr}
(a + b) + c' & = (a + b + c)' & \text{加法定义的第二条规则} \\
             & = (a + (b + c))' & \text{递推假设} \\
             & = a + (b + c)' & \text{加法定义的第二条规则} \\
             & = a + (b + c') & \text{加法定义的第二条规则}
\end{array}
\]

这样我们就证明了加法的结合律。但是加法交换律的证明却并不简单,附录一给出了完整的证明。

%\begin{wrapfigure}{R}{0.4\textwidth}
\begin{figure}[htbp]
\centering
\begin{tikzpicture}[scale=0.8]
\draw (0, 0) rectangle (2, 1)
    (2, 0) rectangle (3, 1);
\draw (0, -1) rectangle (1, -2)
    (1, -1) rectangle (3, -2);
\end{tikzpicture}
\caption{加法交换律的几何证明。将上方的图形倒过来看,或者在镜中看}
\end{figure}
%\end{wrapfigure}

\begin{Exercise}
\begin{enumerate}
\item 定义0的后继为1,证明对于任何自然数都有$a \cdot 1 = a$
\item 证明乘法结合律和交换律
\item 证明乘法分配律
\end{enumerate}
\end{Exercise}

\section{自然数的代数结构}
有了加法和乘法,我们就可以定义更复杂的计算。第一个例子是从零开始的累加:$0 + 1 + 2 + ... $

\be
\begin{array}{l}
sum(0) = 0 \\
sum(n + 1) = (n + 1) + sum(n)
\end{array}
\ee

第二个例子是阶乘$n!$。

\be
\begin{array}{l}
fact(0) = 1 \\
fact(n + 1) = (n + 1) \cdot fact(n)
\end{array}
\ee

比较这两个例子,我们发现它们非常相似。尽管人工智能日新月异地发展,智慧生命和智能机器的一大区别就是能否“跳出系统”,到更高的层次进行抽象。这是我们人类心智中最强大、神秘、复杂、难以捉摸的一部分\cite{GEB}。

我们发现累加和阶乘都有一个针对自然数零起始值,累加始自零,阶乘始自一。针对递归情况,它们都将某一操作应用到一个自然数和它的后继上。累加是$n' + sum(n)$,阶乘是$n' \cdot fact(n)$。如果我们把针对零的起始值抽象为$c$,把递归中的操作抽象为$h$,就可以用一个统一的形式概括累加和阶乘。

\be
\begin{array}{l}
f(0) = c \\
f(n + 1) = h(f(n))
\end{array}
\ee

这是一个在自然数上的递归结构。我们观察一下它在前几个自然数上的表现。

\begin{tabular}{l|l}
$n$ & $f(n)$ \\
\hline
0 & $c$ \\
1 & $f(1) = h(f(0)) = h(c)$ \\
2 & $f(2) = h(f(1)) = h(h(c))$ \\
3 & $f(3) = h(f(2)) = h(h(h(c)))$ \\
... & ... \\
$n$ & $f(n) = h^n(c)$
\end{tabular}

其中,$h^n(c)$表示我们叠加在$c$上重复进行$n$次$h$操作。它是\underline{原始递归}形式的一种(\cite{Bird97},第5页)。更进一步,如果观察此前我们在(\ref{eq:foldn})定义的函数$foldn$,就会发现它们之间的关系:

\be
f = foldn(c, h)
\ee

细心的读者会观察到,我们最初定义的$foldn$带有三个参数,为什么这里只有两个了呢?实际上我们可以写成$f(n) = foldn(c, h, n)$。当我们仅传递给三元函数$foldn$前两个参数,它实际上就成为了接受一个自然数为参数的一元函数了。我们可以这样看待它:$foldn(c, h)(n)$。

我们称$foldn$为自然数上的$fold$操作。令$c$为$zero$,$h$为$succ$,我们就得到了自然数。如同上面的表格,我们可以得到一个序列:

\[
zero, succ(zero), succ(succ(zero)), ... succ^n(zero), ...
\]

如果$c$不是$zero$,$h$不是$succ$,则$foldn(c, h)$就描述了和自然数\underline{同态}(homomorphism)的某种事物。我们来看几个例子:

\[
(+ m) = foldn(m, succ)
\]

这个例子描述了将自然数$n$增加$m$的操作,将它依次作用到自然数上可以产生和自然数同构的序列$m, m + 1, m + 2, ..., n + m, ...$

\[
(\cdot m) = foldn(0, (+ m))
\]

这个例子描述了将自然数$n$乘以$m$的操作,将它依次作用到自然数上可以产生和自然数同构的序列$0, m, 2m, 3m, ..., nm, ...$

\[
m^{()} = foldn(1, (\cdot m))
\]

这个例子描述了对自然数$n$取$m$次幂的操作,将它依次作用到自然数上可以产生和自然数同构的序列$1, m, m^2, m^3, ..., m^n, ...$

那么,我们思考出的这个抽象工具$foldn$能否描述累加和阶乘呢?我们观察下面的这个表格:

\begin{tabular}{r|l|l|l|l|l|l}
$n$ & 0 & 1 & 2 & 3 & ... & $n'$ \\
\hline
$sum(n)$ & 0 & 1 + 0 = 1 & 2 + 1 = 3 & 3 + 3 = 6 & ... & $n' + sum(n)$ \\
\hline
$n!$ & 1 & 1 $\times$ 1 = 1 & 2 $\times$ 1 = 2 & 3 $\times$ 2 = 6 & ... & $n' \cdot (n!)$
\end{tabular}

这里的关键问题是$h$必须是个二元操作,它能够对$n'$和$f(n)$进行运算。为此,我们将$c$也定义为一个二元数$(a, b)$\footnote{在计算机程序中,也称为二元组(tuple),或者对(pair)。}。然后针对二元数$(a, b)$定义某种类似“succ”的操作。最终为了获取结果,我们还需要要定义从二元数中抽取$a$和$b$的函数:

\be
\begin{array}{l}
1st (a, b) = a \\
2nd (a, b) = b
\end{array}
\ee

这样我们就可以定义累加和阶乘了。首先是累加的定义:

\[
\begin{array}{lr}
c = (0, 0) & \text{二元数的起始值} \\
h (m, n) = (m', m' + n) & \text{第一个数取后继,第二个数加第一个数的后继} \\
sum = 2nd \cdot foldn(c, h) \\
\end{array}
\]

我们看看,从起始值$(0, 0)$开始,会怎样一步一步递推出累加的结果。

\begin{tabular}{r|l|l}
$(a, b)$ & $(a', b') = h (a, b)$ & $b'$\\
\hline
(0, 0) & (0 + 1 = 1, 1 + 0 = 1) = (1, 1) & 1 \\
(1, 1) & (1 + 1 = 2, 2 + 1 = 3) = (2, 3) & 3 \\
(2, 3) & (2 + 1 = 3, 3 + 3 = 6) = (3, 6) & 6 \\
... & ... & ... \\
$(m, sum(m))$ & $(m + 1, m + 1 + sum(m))$ & $sum(m + 1)$
\end{tabular}

类似地,我们用$foldn$定义出阶乘。

\[
\begin{array}{lr}
c = (0, 1) & \text{阶乘的起始值} \\
h (m, n) = (m', m'n) & \text{阶乘的递推} \\
fact = 2nd \cdot foldn(c, h) \\
\end{array}
\]

在累加和阶乘的定义中,我们使用了符号“$\cdot$”点来连接两个函数$2nd$和$foldn(c, h)$。我们称之为函数组合,$f\cdot g$表示先将函数$g$应用到变量上,然后再将函数$f$应用到结果上。即$(f\cdot g)(x) = f(g(x))$。

\begin{wrapfigure}{R}{0.4\textwidth}
%\begin{figure}[htbp]
 \centering
 \includegraphics[scale=0.4]{img/Fibonacci.eps}
 \caption{比萨的列奥纳多,又称斐波那契(Leonardo Pisano, Fibonacci),1175年-1250年}
 \label{fig:abstract-num}
%\end{figure}
\end{wrapfigure}

为了展示这一抽象工具的强大,我们再来看一个例子:斐波那契数列。这一数列是用中世纪数学家比萨的列奥纳多命名的。斐波那契来自拉丁文filius Bonacci意思是波那契之子。斐波那契的父亲当时是商人,在北非以及地中海一带经商,斐波那契逐渐向当地阿拉伯人学习了印度——阿拉伯的数字系统并通过他的著作《算盘书》(Liber Abaci)将其介绍到欧洲。中世纪的欧洲一直使用罗马数字系统,我们今天在一些钟表盘上仍然可以看到罗马数字。例如2018年的罗马数字表示为MMXVIII,其中一个M代表1000,两个M代表2000,X代表10,V表示5,三个I表示3。把这些加起来得到2018。斐波那契引入欧洲的是我们今天仍然使用的位值制十进制系统。它使用了印度数学发明的零,不同的数字在不同的位置上含义不同。这一先进的数字系统极大地方便了计算,广泛应用在记账、利息、汇率等方面。《算盘书》更是对欧洲的数学复兴产生了深远的影响。

斐波那契数列来自《算盘书》中的一个问题:兔子在出生两个月后,就有繁殖能力,一对兔子每个月能生出一对小兔子来。如果所有兔都不死,那么一年以后可以繁殖多少对兔子?开始时有一对兔子。第一个月小兔尚未具备繁殖能力,所以仍然只有一对兔子。第二个月它们生下一对小兔,共有两对。第三个月大兔子又生下一对小兔,而上月生的小兔还在成长,总共有2+1=3对。第四个月有两对大兔子产下两对小兔,加上原有的三对兔子,总共有3+2=5对。按照这样,我们可以得到一个序列

\{1, 1, 2, 3, 5, 8, 13, ...\}

这个数列很有规律,从第三项后,任何一项都等于前量项的和。
TODO:Fibonacci数列的介绍和定义

\be
\begin{array}{l}
F_0 = 0 \\
F_1 = 1 \\
F_{n+2} = F_n + F_{n+1}
\end{array}
\ee

TODO: 增加注脚说明我们以后会给出一个基于无穷序列的斐波那契数列的定义

\be
\begin{array}{l}
F = 1st \cdot foldn((0, 1), h) \\
h (m, n) = (n, m + n)
\end{array}
\ee


TODO: 群,monoid和半群的概念

\begin{Exercise}
\begin{enumerate}
\item 使用$foldn$定义$()^m$,它计算给定自然数的$m$次幂。
\end{enumerate}
\end{Exercise}

\section{自然数程序}
使用自然数的代数结构的程序的例子

将一类程序的结构抽象为代数结构

使用自然数的代数结构抽象更多的程序

\ifx\wholebook\relax \else
\begin{thebibliography}{99}

\bibitem{wiki-number}
Wikipedia. ``古代计数系统的历史''. \url{https://en.wikipedia.org/wiki/History_of_ancient_numeral_systems}

\bibitem{trip-to-number-kindom}
[美]卡尔文$\cdot$C$\cdot$克劳森. ``数学旅行家:漫游数王国''. 袁向东、袁钧译,上海教育出版社。ISBN:

\bibitem{wiki-babylonian-num}
Wikipedia. ``古巴比伦数字''. \url{https://en.wikipedia.org/wiki/Babylonian_numerals}

\bibitem{GEB}
[美]候世达 ``哥德尔、埃舍尔、巴赫——集异壁之大成''. 商务印书馆 1996. ISBN: 978-7-100-01323-9

\bibitem{Bird97}
Richard Bird, Oege de Moor. ``Algebra of Programming''. Univerisity of Oxford, Prentice Hall Europe. 1997. ISBN: 0-13-507245-X.

\end{thebibliography}

\expandafter\enddocument
%\end{document}

\fi
