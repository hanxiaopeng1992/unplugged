\ifx\wholebook\relax \else

\documentclass[UTF8]{article}

\input{../common-zh-cn.tex}

\setcounter{page}{1}

\begin{document}

% ================================================================
%                 Digit lock
% ================================================================

\title{自然数}

\author{刘新宇
\thanks{{\bfseries 刘新宇} \newline
  Email: liuxinyu95@gmail.com \newline}
  }

\maketitle
\fi

\markboth{自然数}{计算机程序的代数结构}

\section{数的诞生}
数的概念是伴随着我们人类进化产生的。数的产生甚至可能在语言之前。有人认为,数直接催生了人类的语言和文字。我们的祖先在长期的狩猎——采集活动中,逐渐掌握了数的概念。
最早可能是简单的数数,例如数清果实的数量。随着文明的发展,人们逐渐开始进行物品交易。随着交易物品数量增长,很快需要采用助记工具来处理更大的数。
考古发现在今天的伊朗一代,人们在公元前4000年开始使用陶制的小球来辅助计数。比如用两个刻有十字的小球代表两只羊,同时还有代表十只羊,二十只羊的不同小球。
为了防止忘记或者篡改数过的数目,人们还把这些小球放在陶罐中用泥土封存起来。图\ref{fig:clay-token}是乌鲁克(Uruk)时期的陶制计数罐和小球\cite{wiki-number}。
交易过程中,人们可以通过这些工具掌握货物的数量\cite{trip-to-number-kindom}。

\begin{figure}[htbp]
 \centering
 \includegraphics[scale=0.5]{img/clay-envelope.eps}
 \caption{卢浮宫陈列的乌鲁尔时期的计数陶罐和一组计数陶球。}
 \label{fig:clay-token}
\end{figure}

随着交易的增加和数目的增大,这样的陶罐和小球就不够方便了。大约公元前3500年,美索不达米亚的苏美尔人开始在泥板上刻化符号来纪录交易。将泥板烤硬后就可以方便的保存纪录。人们在这一时期用坚硬的笔在泥板上刻出不同的符号,同时表示交易的物品和数量。

更大进步发生在公元前3100年,从出土的泥板中,我们发现苏美尔人开始将数字从它代表的物品中抽象出来。人们不再使用一个符号同时表示物品和数量,而是用一个符号表示交易的数量,接下来用另一个符号表示交易的物品。这些泥板上的符号逐渐演变成了古巴比伦的楔形文字。

\begin{figure}[htbp]
 \centering
 \includegraphics[scale=0.8]{img/Babylonian_numerals.eps}
 \caption{巴比伦楔形文字中的数字\cite{wiki-babylonian-num}。}
 \label{fig:clay-token}
\end{figure}


数对具体事物的抽象

\section{皮亚诺自然数公理}

\section{自然数的代数结构}
具体例子
群,monoid和半群的概念

\section{自然数程序}
使用自然数的代数结构的程序的例子

将一类程序的结构抽象为代数结构

使用自然数的代数结构抽象更多的程序

\ifx\wholebook\relax \else
\begin{thebibliography}{99}

\bibitem{wiki-number}
Wikipedia. ``古代计数系统的历史''. \url{https://en.wikipedia.org/wiki/History_of_ancient_numeral_systems}

\bibitem{trip-to-number-kindom}
[美]卡尔文$\cdot$C$\cdot$克劳森. ``数学旅行家:漫游数王国''. 袁向东、袁钧译,上海教育出版社。ISBN:

\bibitem{wiki-babylonian-num}
Wikipedia. ``古巴比伦数字''. \url{https://en.wikipedia.org/wiki/Babylonian_numerals}
\end{thebibliography}

\expandafter\enddocument
%\end{document}

\fi
